\documentclass{amsart}
\usepackage{complexity}
\setlength{\parindent}{0pt}

\setlength{\textwidth}{\paperwidth}
\addtolength{\textwidth}{-2in}
\calclayout

\newtheorem{theorem}{Theorem}[section]
\newtheorem{corollary}{Corollary}[theorem]
\newtheorem{lemma}[theorem]{Lemma}

\theoremstyle{definition}
\newtheorem{definition}[theorem]{Definition}
\newtheorem{example}[theorem]{Example}
\newtheorem{proposition}[theorem]{Proposition}
\newtheorem{conjecture}[theorem]{Conjecture}
\newtheorem{xca}[theorem]{Exercise}

\newcommand{\I}{\mathcal{I}}
\newcommand{\MaxCNF}{\mathsf{Max3CNF}}
\newcommand{\ELin}{\mathsf{MaxE3Lin}}
\newcommand{\MaxCSP}{\mathsf{MaxCSP}(\Psi)}
\newcommand{\MaxCut}{\mathsf{MaxCut}}
\newcommand{\GapMaxCSP}[2]{\mathsf{Gap_{#1,#2}}\MaxCSP}
\newcommand{\GWConstant}{\frac{\cos^{-1}(\rho)}{\pi}}
\newcommand{\UG}{\mathsf{UG}}
\newcommand{\GapDeltaULC}{\mathsf{Gap}_{\delta, 1-\delta}\ULC}
\newcommand{\GapMaxCut}{\mathsf{Gap}_{\GWConstant + \epsilon, \frac{1-\rho}{2} - \epsilon}\mathsf{MaxCut}}
\newcommand{\ULC}{\mathsf{UniqueLC}(m)}
\newcommand{\GapULC}{\mathsf{GapUniqueLC}(m)}
\newcommand{\LC}{\mathsf{Gap}_\epsilon\mathsf{LC}}


\theoremstyle{remark}
\newtheorem*{remark}{Remark}
\newtheorem*{note}{Note}
\numberwithin{equation}{section}

\theoremstyle{remark}
\newtheorem{notation}[theorem]{Notation}
\begin{document}

\renewcommand{\phi}{\varphi}
%\newcommand{\Max3CNF}{\mathsf{Max3CNF}}

\title{Notes on PCPs and Unique Games}

\author{Edward Kim}
\email{ehkim@cs.unc.edu}

\begin{abstract}
We expound on the basic theory of approximating $\NP$-hard problems. Starting with the PCP Theorem, we introduce background for the Unique Games Conjecture (UGC) and the optimality of the Goemans-Williamson Algorithm for MAXCUT assuming the conjecture is true.
%Finally, we make some connections to Quantum Information Theory, especially Quantum PCPs (qPCPs).
\end{abstract}

\maketitle
\tableofcontents

\chapter{Introduction}
\label{sec:intro}

One of the most widely-used techniques for performing safety analysis of non-linear dynamical systems is reachable set computation.
%
For instance, reachability analysis has found a panoply of applications in formally verifying the safety properties of Cyber-physical Systems (CPS), such as autonomous vehicles \cite{althoff2010reachability}, F-16 aircraft \cite{heidlauf2018verification}, and CPS systems governed by Neural Network Controllers \cite{tran2019star, fan2020reachnn, bak2021nnenum}.
%
The reachable set is defined to be the set of states visited by at least one of the trajectories of the system starting from an initial set and propagated forward in time by a fixed number of steps.
%
Computing the exact reachable set for non-linear systems is challenging due to several reasons:
%
First, unlike linear dynamical systems whose solutions can be expressed in closed form, non-linear dynamical systems generally do not admit such a nice form.
%
Second, computationally speaking, current tools for performing non-linear reachability analysis are not very scalable. This is also in stark contrast to several scalable approaches developed for linear dynamical systems \cite{duggirala2016parsimonious, bak2017simulation}.
%
Finally, computing the reachable set using various set representations involves wrapping error which may be too conservative for practical use.
%
That is, the over-approximation acquired at a given step would increase the conservativeness of the over-approximation for all future steps.

One of the several techniques for computing the over-approximation of reachable sets for discrete non-linear systems is encoding the reachable set through parallelotope bundles.
%
Here, the reachable set is represented as a parallelotope bundle, an geometric data structure representing an intersection of several simpler objects called parallelotopes.
%
One of the advantages of this technique is its exploitation of a special form of non-linear optimization problem to over-approximate the reachable set.
%
The usage of a specific form of non-linear optimization mitigates many drawbacks involved with the scalability of non-linear analysis.

However, wrapping error still remains to be a problem for reachability using parallelotope bundles.
%
An immediate reason stems from the responsibility of the practitioner to define the template directions specifiying the parallelotopes.
%
Often times these template directions are selected to be either the cardinal axis directions or some directions from octahedral domains.
%
However, it is not certain that the axis-aligned and octagonal directions are optimal for computing reachable sets over general non-linear dynamics.
%
Additionally, even an expert user of reachable set computation tools may not be able to ascertain a suitable set of template directions for computing reasonably accurate over-approximations of the reachable set.
%
Picking unsuitable template directions would only cause the wrapping error to grow, leading to the aforementioned issue of overly conservative reachable sets.

In this thesis, we investigate techniques for generating template directions automatically and dynamically, which is the culmination of several publications in different venues \cite{kim2020kaa,kim2021automatic,geretti2021arch}.
%
Specifically, we propose a method where instead of the user providing the template directions to define the parallelotope bundle, he or she specifies the number of templates whose template directions are to be generated by our algorithm automatically.

To this end, we study two techniques for generating the said template directions.
%
First, we compute a local linear approximation of the non-linear dynamics and use the linear approximation to compute the template directions.
%
Second, we generate a set of trajectories sampled from within the reachable set and use Principal Component Analysis (PCA) over these trajectories.
%
We observe that the accuracy of the reachable set can be drastically improved by using templates generated using these two techniques.
%
To address scalability, we demonstrate that even when the size of the initial set increases, our template generation algorithm returns more accurate reachable sets than both manually-specified and random template directions.
%
We experiment with our dynamic template generation algorithm's effectiveness on approximating the reachable set of high-dimensional COVID19 dynamics proposed by the Indian Supermodel Committee \cite{indiansuper2020supermodel}. The results were published in an ACM blogpost detailing the utility of reachable set computation in modeling disease dynamics \cite{bak2021covid}.

Finally, we investigate an application of Bernstein expansion-based reachability to the real-time domain.
%
We attempt to pre-compute the relevant Bernstein coefficients over the entire domain and prune the coefficients which do not appear as either a maximum or minimum coefficient.
%
The idea is to decrease the total number of coefficients the reachability algorithm has to compute in order to gain a speedup.
%
However, we show that there are several obstacles which hinder the utility of our pre-processing step.
% [Introduction para]asdf
% \begin{itemize}
% \item Reachable set computation.
% \item non-linear dynamics is challenging.
% \item Overapproximation, wrapping error.
% \item Parallelotope bundle reachability.
% \end{itemize}

% [Templates for reachability]
% \begin{itemize}
% \item Often the templates are generated statically.
% \item However, the most appropriate directions for templates that improve the accuracy is unknwon.
% \item Challenges: Wrapping error, cannot predict.
% \item Static template directions often aggrevate such errors.
% \item Generating templates dynamically is a challenge.
% \end{itemize}

% [Template generation]
% \begin{itemize}
% \item Template based overapproximation are used extensively in verification.
% \item More on this in the related work section.
% \item Dynamic templates using PCA have been investigated already.
% \item Not explicitly in the context of parallelotope bundles.
% \item In this paper we propose two techniques for  generating templates dynamically.
% \end{itemize}

\section{Related Work}
\label{sec:related}

Reachable set computation of non-linear systems using template polyhedra and Bernstein polynomials was first proposed in~\cite{dang2009image}.
%
In~\cite{dang2009image}, Bernstein polynomial representation is used to compute an upper bound of a special type of non-linear optimization problem.
%
This enclosing property of Bernstein polynomials has been actively studied in the area of global optimization \cite{nataray2002algorithm, garloff2003bernstein, nataraj2007new}.
%
Furthermore, several heuristics have been proposed for improving the computational performance of optimization using Bernstein polynomials~\cite{smith2009fast,munoz2013formalization}.

Several improvements to this algorithm were suggested in~\cite{dang2012reachability, sassi2012reachability} and~\cite{dang2014parameter} extends it for performing parameter synthesis.
%Hey
The representation of parallelotope bundles for reachability was proposed in~\cite{dreossi2016parallelotope} and the effectiveness of using bundles for reachability was demonstrated in~\cite{dreossi2017sapo, dreossi2017reachability}.
%
However, all of these papers used static template directions for computing the reachable set.
%
In other words, the user must specify the template directions before the reachable set computation proceeds.

Using template directions for reachable set has been proposed in~\cite{sankaranarayanan2008symbolic} and later improved in~\cite{dang2011template}.
%
Leveraging the Principal Component Analysis of sample trajectories for computing reachable set has been proposed in~\cite{stursberg2003efficient,chen2011choice,seladji2017finding}.
%
More recently, connections between optimal template directions for reachability of linear dynamical systems and bilinear programming have been highlighted in~\cite{gronski2019template}.
%
For static template directions, octahedral domain directions~\cite{clariso2004octahedron} remain a popular choice.

\section{Approximation Algorithms and CSPs}
%3.2 Limits of Approximation – Introduction (TIFR)
%AB - Chapter 11
%Explain intuition.
\subsection{Definitions and Examples}
The theory of approximating $\NP$-hard problems roots itself in the following question: ``Is it possible to efficiently approximate $\NP$-complete problem to some arbitrary degree of accuracy?" Since Cook-Levin result demonstrated that the $\SAT$ decision problem is $\NP$-complete, the question could be rephrased as ``Can we find an efficient algorithhm for $\SAT$ which outputs an assignment which satisfies a $1-\delta$ fraction of the clauses for any constant $\delta >  0$?" To this end, let us first introduce a few motivating examples which will find utility in our analysis. \newline

Consider $\mathsf{Max3CNF}$, the problem of, given a 3CNF formula $\phi$ as input, outputting an assignment which maximizes the number of clauses satisfied in $\phi$. Given any assignment $x$ to the variables, say as an $n$-bit string while viewing $\phi:\{0,1\}^n \rightarrow \{0,1\}$, there is some $0 \leq \rho_{\phi,x} \leq 1$ representing the fraction of clauses satisfied in $\phi$. Take $\mathsf{Opt}(\phi)$ to be the maximum such value over all such assignments:

\[ \mathsf{Opt}(\phi) = \max_{x \in \{0,1\}^n}\rho_{\phi,x} \]

Naturally, $\mathsf{Max3CNF}$ is $\NP$-hard since its corresponding decision problem $\mathsf{3CNF}$ is $\NP$-complete i.e $\phi$ is satisfiable iff $\mathsf{Opt}(\phi) = 1$.
%
However, we can ask if there exists an algorithm $A$ which outputs an assignment which satisfies at least some $\beta\cdot\mathsf{Opt}(\phi)$ fraction of the clauses of $\phi$ for some $\beta \leq 1$. To formalize this: \newline
%
\begin{definition} \label{Opt3SATDef}
For $\beta \leq 1$, a polynomial-time algorithm $A$ is deemed as a $\beta$-approximation algorithm for $\mathsf{Max3CNF}$ if $A(\phi)$ outputs an assignment which satisifies at least $\beta\cdot\mathsf{Opt}(\phi)$ fraction of $\phi$'s clauses for every 3CNF instance $\phi$. More specifically, the algorithm $A$ outputs some value such that:
\begin{equation}
  \beta\cdot\mathsf{Opt}(\phi) \leq A(\phi) \leq \mathsf{Opt}(\phi)
\end{equation}
If $\mathsf{Opt}(\phi) = 1$, then $\phi$ is said to be \emph{satisfiable}.
\end{definition}

A canonical example of a simple approximation algorithm for $\mathsf{Max3CNF}$ is the following scheme: for every variable in $\phi$ choose with uniform probability an assignment from $\{0,1\}$. The probability of any given clause being satisfied by such a random assignment is $\frac{7}{8}$. Thus
%
\[ \mathbb{E}_{x \in \{0,1\}^n}[\text{\# satisfied clauses of $\phi(x)$}] = \frac{7m}{8} \]
%
where $m$ is the number of satisfiable clauses of $\phi$. This gives us a $\frac{7}{8}$-approximation algorithm for $\mathsf{Max3CNF}$. Here's another example:
%
\begin{example} \label{e3linexample}
  Let $\mathsf{MaxE3Lin}$ define the problem of the following form: let $\xi$ be defined as a system of linear equations over the field $\mathbb{F}_2$ where every equation contains \emph{exactly} $3$ variables from $n$ variables $x_1,\cdots,x_n$. Find an assignment to the variables which maximizes the number of satisified linear equations in $\xi$. An example is the following system over variables $x_1,x_2,x_3,x_4$:
  %
  \begin{equation*}
    \begin{alignedat}{3}
      x_1 & +{} &  x_2 & +{} & x_3 & = 0 \\
      x_1 & +{} &  x_2 & +{} &  x_4 & = 1 \\
      x_1 & +{} & x_3 & +{} & x_4 & = 1
\end{alignedat}
\end{equation*}
\end{example}
%
The $\mathsf{E3Lin}$ represents a linear system of $\mathsf{E}$xactly $\mathsf{3}$ variables. We can extend the notation treated above for $\mathsf{Max3CNF}$ to this situation. In other words, if $\rho_{\xi,x}$ is the fraction of satisfied linear constraints of $\xi$ under assignment $x \in \mathbb{F}_2^n$.
%
\[\mathsf{Opt}(\xi) =\max_{x \in \{0,1\}^n}\rho_{\xi,x}\]
%
A comment regarding this problem. Firstly, if the given E3Lin instance has a guaranteed solution i.e $\mathsf{Opt}(\phi) = 1$, then Gaussian Elimination will always output an assignment which satisfies all linear constaints in polynomial-time. It's more interesting to consider instances where no such solution satisfing all of the constraints exists over $\mathbb{F}_2^n$. In these cases, $\mathsf{Opt}(\xi) = 1 - \epsilon$ for some $\epsilon \leq 1$. This problem also has a fairly simple $\frac{1}{2}$-approximation algorithm: Set all $x_1 = \cdots = x_n  = 0$ or $x_1 = \cdots = x_n  = 1$ depending on which satisfies the most constraints. This scheme must satisfy at least $\frac{1}{2}$ of the constraints for any instance $\xi$. \newline

\begin{example} \label{maxcutexample}
The problem $\MaxCut$ will be casted as follows: given an undirected graph $G=(V,E)$, find the largest cut of $G$. There is a straightforward LP formulation for this problem:

\begin{align}
  &  \min \sum_{(u,v) \in E} e_{u,v} \\
  & e_{u,v} \geq
  \max \begin{cases}
      x_1 + x_2 \\
      2 - (x_1 + x_2)
  \end{cases} \\
  & e_{u,v} \in [1,2] \\
  & x_v \in [0,1]
\end{align}
%
where the $e_{u,v}$ variables represent the edges $e \in E$ of the input graph and the $x_v$ represent the vertex variables $v \in V$. This will actually be a relaxation from its respective integer program in which the variables can take the integer values:

\begin{align}
  e_{u,v} = \begin{cases}
              1 \text{ if } e_{u,v} \in \mathcal{C} \\
              2 \text{ otherwise }
            \end{cases}
  x_v = \begin{cases}
              0 \text{ if } v \text{ is in partition } S \\
              1 \text{ otherwise }
        \end{cases}
\end{align}

where $\mathcal{C} \subseteq E$ is the edges constituting the cut and $(S,\bar{S})$ is the partition defining the cut. Naturally, a solution to the Integer program will also be a solution to the LP above, which implies that

\[ \mathsf{Opt}_{LP} \geq \MaxCut(G)\]

There actually exists an Semi-definite Programming (SDP) relaxation for $\MaxCut$ called the \emph{Goemans-Williamson} algorithm which will be treated in Section \ref{goemans}.

\end{example}

\subsection{Constraint Satisfaction Problems}

 Generally speaking, it's useful to concretize these problems into a unified framework. The examples presented in the last section have some common interpretation shared between them: they could all be seen as manifestations of Constaint-Satisfaction Problem (CSP):
%
\begin{definition}
  Let $\Omega$ be a finite set deemed as the \emph{domain}. A constraint satisfaction problem (CSP) $\Psi$ over domain $\Omega$ is a finite set of predicates $\psi:\Omega^r \rightarrow \{0,1\}$ where $r$ would be the \emph{arity} of predicate $\psi$. The predicates can be of different arities. The arity of the CSP $\Psi$ would be the maximum of all the arities of the predicates in $\Psi$. \newline

  \noindent An \emph{instance} $\mathcal{I}$ over CSP $\Psi$ is a set of tuples $(S,\psi)$ where if $r$ is the arity of predicate $\psi$, $S = (v_1,\cdots,v_r)$ is some ordered tuple of variables taken from finite set $V$ consisting of variables contained in CSP $\Psi$. These tuples are called the \emph{constaints} of $\mathcal{I}$. In addition, we add the conditon that every variable show up in at least one constraint. Variables which do not appear in any of the constraints can simply be removed. \newline

  \noindent Now given some instance $\mathcal{I}$, an \emph{assignment} $F:V \rightarrow \Omega$ is simply some  map between the variables and the domain. We say that $F$ satisfies a constaint $(S,\psi)$ if $\psi(F(S)) = 1$. For tuple $S = (v_1, \cdots, v_r)$, we define $F(S) = (F(v_1), \cdots, F(v_r))$. Consider the fraction of constraints of satisfied by $F$ in $\mathcal{I}$:
  %
  \begin{equation}
  \mathsf{Val}_{\I}(F) = \mathbb{E}_{(S, \psi) \sim \I}[\psi(F(S))]
  \end{equation}
  By taking the maximum fraction over all such assignments:
  %
  \begin{equation}
  \mathsf{Opt}(\I) = \max_{F: V \rightarrow \Omega} \mathsf{Val}_{\I}(F)
  \end{equation}
\end{definition}

  %
  \begin{example} \hfill{}
    \begin{itemize}
      \item For $\MaxCNF$, the domain would be $\Omega=\{0,1\}$ and the CSP $\Psi$ would be composed of the predicate $\vee_{3}:\{0,1\}^3 \rightarrow \{0,1\}$ just taking the logical ORs of the three input variables. Any instance $\phi$ would be composed of $(S,\vee_{3})$ where $S$ would be the variables inputted into $\vee_{3}$. Hence, an assignment $F$ would be an assignment into the variables found in 3CNF $\phi$, showing that $\mathsf{Opt}(\phi)$ aligns with the definition given in the last section. \newline

      \item For $\ELin$, the domain would be $\Omega = \mathbb{F}_2$ and  $\Psi$ consist of predicates of the form $(x_1,\cdots,x_3) = x_1 + x_2 + x_3$ and $(x_1,\cdots,x_3) = x_1 + x_2 + x_3 + 1$ representing both types of linear constraints found in an system of three-variable equations over $\mathbb{F}_2$. An instance $\xi$ would consist of constraints $(S,\psi)$ where $S$ would be a three variable tuple containing the variables showing up in the linear constraint $\psi$. \newline

      \item For $\mathsf{MaxCut}$, the domain can be defined as $\Omega = \{-1,1\}$ with $\Psi$ set to the simple predicate $\neq:\{-1,1\}^2 \rightarrow \{0,1\}$. This simply tests if the inputted values are not equal. The variable set $V$ of an instance $\I$ would be indexed by the vertices contained in a graph $G=(V,E)$. There is one constraint tuple, $((v_i,v_j), \neq)$ for every edge $(v_i,v_j) \in E$. Thus, an assignment $F:V \rightarrow \{-1,1\}$ would encode a partition of $V$ into a cut $C$ with a constraint becoming satisified if the corresponding edge is contained in $C$. \newline

      \item The CSP for $\mathsf{Max3Color}$, the maximization counterpart for the $\NP$-complete decision problem, $\mathsf{3Color}$, is similarly defined to that of $\mathsf{MaxCut}$ except our domain would be $\Omega = \{0,1,2\}$. This signifies the three possible colors to color any vertex $v \in V$ in an input graph $G = (V,E)$.


    \end{itemize}
  \end{example}

  As implied in the examples above, there is a generic method to formulate a maximization problem in respect to a given CSP $\Psi$:
  %
  \begin{definition}
    For a given CSP $\Psi$, formulate $\mathsf{MaxCSP}(\Psi)$ as the problem: given an instance $\I$, output an assignment $F$ which satisfies the largest number of constraints in $\I$.
  \end{definition}

\subsection{Gap Problems}

The $\NP$-hardness theory frequently relies on Karp reductions from decision problems to decision problems. In light of this, we can tailor optimization problems to related promise problems in the form of so-called \emph{gap problems}.

\begin{definition} \label{promiseDef}
  A \emph{promise problem} is defined as a tuple $(\mathsf{YES}, \mathsf{NO})$ where $\mathsf{YES},\mathsf{NO} \subseteq \Sigma^*$ in respect to some alphabet $\Sigma$. Furthermore, we require that $\mathsf{YES} \cap \mathsf{NO} = \emptyset$ but not necessarily that $\mathsf{YES} \cup \mathsf{NO} = \Sigma^*$.
\end{definition}

\begin{definition} \label{gapDef}
  Given a $\MaxCSP$ problem, we define $\GapMaxCSP{\alpha}{\beta}$ for $\alpha < \beta$ as the promise problem: given an instance $\I$:
  \begin{align}
      \I \in \mathsf{YES} & \iff \mathsf{Opt}(\I) \geq \beta \\
        \I \in \mathsf{NO} & \iff \mathsf{Opt}(\I) < \alpha
  \end{align}
\end{definition}

The $\NP$-hardness of approximation algorithms reduces to that of gap problems as shown in the below observation:

\begin{theorem} \label{GapCSPtoAlgHard}
Suppose $\GapMaxCSP{\alpha}{\beta}$ is $\NP$-hard for CSP $\Psi$, then approximating $\MaxCSP$ to at least an $\frac{\alpha}{\beta}$ factor is $\NP$-hard.
\end{theorem}
%
\begin{proof}
Suppose there exists an algorithm $A$ which can $\frac{\alpha}{\beta}$-approximate $\MaxCSP$. For an instance $\I$ such that $\mathsf{Opt}(\I) \geq \beta$:

\begin{gather*}
A(\I) \geq  \frac{\alpha}{\beta} \cdot \mathsf{Opt}(\I) = \frac{\alpha}{\beta} \cdot \beta =  \alpha \\
\end{gather*}

Else if $\mathsf{Opt}(\I) < \alpha$
\begin{gather*}
A(\I) \leq \mathsf{Opt}(\I) < \alpha
\end{gather*}
by Definition \ref{Opt3SATDef} adapted to $\MaxCSP$ instances. Hence, the algorithm can decide  $\GapMaxCSP{\alpha}{\beta}$ by checking it's outputted value in respect to $\alpha$.
\end{proof}
%
This in particular demonstrates that showing the hardness of approximating a particular problem is equivalent to showing the hardness of its corresponding gap problem.

%It was shown early on that traditional Karp reductions were unlikely to produce the gaps required in Definition \ref{gapDef} in reducing $\SAT$ to such a promise problem.


%Traditional Karp reductions don't work

\section{The PCP Theorem}
\subsection{Intuitions}
This section will be centered around the seminal PCP Theorem \cite{arora1998proof}, \cite{arora1998probabilistic}, which characterized $\NP$ in a framework considered unconventional at the time. PCPs, or Probabilistically Checkable Proofs, represent a twist on the idea of $\NP$. Recall that $\NP$ roughly represents the languages which have verifiers which can check proofs of membership in polynomial time. PCPs represent an extension of this definition where the verifier can be \emph{probabilistic} and is granted \emph{random access} to the proof string $\pi$. If we allow the verfier to simply query $\pi$ by outputting a index $i$, it has access to $\pi[i]$. Since we can express an index in $\log{n}$ bits, this in theory gives the verifier access to proof strings of exponential length. To formalize these notions, we begin with definitions:

\begin{definition}
  Given a language $L$ and $r,q: \mathbb{N} \rightarrow \mathbb{N}$, a \emph{(r(n),q(n))-$\PCP$-verifier} for $L$ consists of a polynomial-time algorithm $V$ with the following properties: \newline

  \begin{itemize}
    \item For input strings $x \in \{0,1\}^n, \; \pi \in \{0,1\}^{\leq N}$ for $N = q(n)2^{r(n)}$, $V$ makes $r(n)$ coin flips and decides $q(n)$ queries addresses $i_1, \cdots, i_{q(n)}$ of the proof $\pi$. Based on these queries, it outputs $1$ for ``accept" or $0$ for ``reject". \newline

    \item (Completeness) For $x \in L$, there exists some proof $\pi$ such that $V(x,\pi,r) = 1$ for all random coin tosses $r$. In other words:
    %
    \begin{equation}
      \mathbb{P}_{r}[V(x,\pi,r) = 1] = 1
    \end{equation}

    \item (Soundness) For $x \not\in L$, for all proofs $\pi$:
    %
    \begin{equation}
      \mathbb{P}_{r}[V(x,\pi,r) = 1] \leq \frac{1}{2}
    \end{equation}
  \end{itemize}
  Define the class $\PCP(r(n),q(n))$ as the set of languages $L$ which has a $(c\cdot r(n),d\cdot q(n))$-$\PCP$-verifier for some $c,d > 0$.
\end{definition}

\begin{remark}
  Sometimes the completeness criterion is too strong for our purposes (see the comments on $\ELin$ problem). In these cases, we like to denote the class $\PCP_{\beta, \alpha}(r(n),q(n))$ as the languages $L$ which have a $(r(n),q(n))$-$\NP$-verifier such that the completeness and soundness criteria are amended as below: \newline

  \begin{itemize}
    \item (Completeness) For $x \in L$, there exists some proof $\pi$ such that $V(x,\pi,r) = 1$ for all random coin tosses $r$. In other words:
    %
    \begin{equation}
      \mathbb{P}_{r}[V(x,\pi,r) = 1] \geq \beta
    \end{equation}

    \item (Soundness) For $x \not\in L$, for all proofs $\pi$:
    %
    \begin{equation}
      \mathbb{P}_{r}[V(x,\pi,r) = 1] \leq \alpha
    \end{equation}
  \end{itemize}
Here, $\beta$ is the \emph{completeness parameter} while $\alpha$ is the \emph{soundness parameter}. The class introduced in the original definition would thus be denoted as $\PCP_{1,\frac{1}{2}}(r(n),q(n))$. PCP verifiers whose completeness parameter is one ($\beta = 1$) is deemed as \emph{perfectly complete}.
\end{remark}

The PCP Theorem says that $\NP$ is \emph{exactly} the class of PCPs which uses a \emph{logarithmic} number of random bits and a \emph{constant} number of queries.
%
\begin{theorem} \label{pcptheorem} (The $\PCP$ Theorem \cite{arora1998proof}, \cite{arora1998probabilistic})
%
\begin{equation}
\NP = \PCP_{1,\frac{1}{2}}(O(\log{n}), O(1))
\end{equation}
\end{theorem}

Actually, one direction of this theorem is not too difficult to see:

\begin{proposition}
For every constants $Q \in \mathbb{N}, c > 0$, $\PCP_{1,\frac{1}{2}}(c\cdot\log{n}, Q) \subseteq \NP$
\end{proposition}

\begin{proof}
Begin with the observation that $\PCP_{1,\frac{1}{2}}(r(n), q(n)) \subseteq \NTIME(q(n)2^{r(n)})$. This is justified by the view of an $\NTIME$ machine simulating the verifier by trying all possible coin tosses and queries to the input string $x$ and proof string $\pi$. It can then count all of the accepting paths to determine the probability of acceptance. If $q = O(1)$ and $r = O(\log{n})$, then the right side of the inclusion will be $\NTIME(2^{O(\log{n})}) = \NP$.
\end{proof}

\begin{remark}
  The queries a PCP verifier makes could be \emph{adaptive} or \emph{non-adaptive}. Adaptive queries can be dependent on the outcome of previous queries while non-adaptive queries are independent of one another. The verifiers in these notes will all be non-adaptive for the sake of presentation. The $\PCP$ Theorem still holds when the verifier makes adaptive queries. The only change would be that the proof length would be at most $2^{r(n) + q(n)}$ rather than at most $q(n)2^{r(n)}$.
\end{remark}


%
\subsection{Equivalence of PCP Theorems}
It may be difficult to understand the importance of the PCP Theorem in its form presented in Theorem \ref{pcptheorem}. It turns out there are other equivalent forms of the PCP Theorem more palatable in the context of our goal to prove hardness of approximation results.

\begin{theorem} \label{pcpgapsat}($\PCP$ Theorem: $\mathsf{Gap3SAT}$-hardness)
The problem $\mathsf{Gap}_{\alpha,1}\mathsf{Max3SAT}$ is $\NP$-hard. In other words, for every $\NP$ language $L$, there exists a polynomial-time reduction $f$ mapping $L$ to 3CNF formulas such that:

\begin{align*}
  x \in L & \implies \mathsf{Opt}(f(x)) = 1 \\
  x \not\in L & \implies \mathsf{Opt}(f(x)) \leq \frac{1}{2}
\end{align*}
\end{theorem}

An immediate consequence of Theorem \ref{pcpgapsat} and Theorem \ref{GapCSPtoAlgHard} is that if there exists an $\alpha$-approximation algorithm for $\mathsf{Max3SAT}$, then $\P = \NP$. With this, we have the first steps towards an inapproximability result: if $\P \neq \NP$, there exists no efficient $\mathsf{Max3SAT}$ algorithm which can approximate better than an $\alpha$ factor. Note that we haven't actually found a concrete value for $\alpha$ yet. This will be addressed once we prove H\aa stad's 3-bit PCP for $\NP$ in a future section.


\begin{theorem} \label{pcptheoremgapcsp}  ($\PCP$-Theorem: $\mathsf{GapMaxCSP}$- hardness) For some constants $q \in \mathbb{N}$, the problem $\mathsf{Gap}_{\alpha,1}\mathsf{Max}$-$\mathsf{qCSP}$ is $\NP$-hard. To elaborate, for every $\NP$ language $L$, there exists a polynomial time reduction mapping an $L$ to a instance $f(x)$ of some CSP $\Psi$ where $\Psi$ consists of $q$-ary predicates, such that

\begin{align*}
  x \in L & \implies \mathsf{Opt}(f(x)) = 1 \\
  x \not\in L & \implies \mathsf{Opt}(f(x)) \leq \frac{1}{2}
\end{align*}
\end{theorem}

\begin{theorem}
  All the PCP Theorems above are equivalent to each other.
\end{theorem}

Before we embark on the proof, let us establish an equivalence between PCPs and CSPs:

\begin{lemma} (Equivalence between PCPs and CSPs)
  Theorem \ref{pcptheorem} and  Theorem \ref{pcptheoremgapcsp} are equivalent.
\end{lemma}

\begin{proof}
First, assume $\NP = \PCP_{1,\frac{1}{2}}(O(\log{n}), O(1))$. We will outline a procedure to convert the verifier $V$ into an instance $\I$ for a $q$-ary CSP $\Psi$ for some constant $q$. For some input string $x \in \{0,1\}^n$ and proof string $\pi$, let $r \in \{0,1\}^{c \cdot \log{n}}$ be the coin flips made by $V$ and $V_{x,r}$ be the deterministic procedure which is executed on input $x$ and coin flip $r$ such that $V_{x,r} = 1$ iff $V$ accepts proof $\pi$ on input $x$ and coin flip $r$. We can define the domain of our constructed CSP $\Psi$ to be $\Omega=\{0,1\}$ and the predicates to be $\{V_{x,r}\}_{r}$. Now our instance $\I$ of $\Psi$ is casted as the tuples $(S, V_{x,r})$ where $S$ will be at most a $q$-sized tuple indicating which indices of the proof $\pi$ are queried when conditioned on $r$. This yields a polynomially-sized $\mathsf{qCSP}$ instance $\I$. Furthermore, since the verifier $V$ runs in polynomial time, it's execution can be simulated on all $r$ to output the instance $\I$ in polynomial-time. Thus, we have given a polynomial-time reduction from an input $x$ to its corresponding CSP instance $\I$, so Theorem \ref{pcptheoremgapcsp}. \newline

Conversely, suppose we had a reduction from $\NP$ to $\mathsf{Gap}_{\alpha,1}\mathsf{Max}$-$\mathsf{qCSP}$ as stated in Theorem \ref{pcptheoremgapcsp}. We devise polynomial-time reduction taking an instance $\mathsf{Max}$-$\mathsf{qCSP}$ to a polynomial-time PCP verifier $V$ using logarithmic number of random bits and a constant number of queries to the supplied proof $\pi$. For an input $x \in \{0,1\}^n$, the proof will be expected to be an assignment to its respective instance $f(x)$ in the notation utilized in Theorem \ref{pcptheoremgapcsp}. Verifier $V$ makes coin flips $r \in \{0,1\}^{c\log{n}}$ to choose one constraint tuple $(S,\psi)$ where $\psi$ is some $q$-ary predicate. Only a logarithmic number of random bits are required to query any constraint in instance $f(x)$ as the polynomial-time $\NP$ reduction can only generate a polynomial number of such constraints.  The PCP only has to make $q$-queries to the proof $\pi$ to find the assignments to the variables listed in $S = (v_1,\cdots,v_q)$. By the properties listed in Theorem \ref{pcptheoremgapcsp}, the PCP verifier $V$ must have completeness $1$ and soundness $\leq \frac{1}{2}$ as claimed.
\end{proof}

\begin{lemma} (Equivalence between $\mathsf{Gap}$-$\mathsf{3SAT}$ and $\mathsf{Max}$-$\mathsf{qCSP}$). Theorem \ref{pcpgapsat} and Theorem \ref{pcptheoremgapcsp} are equivalent.
\end{lemma}
%
\begin{proof}
To be written...
\end{proof}

%2 The PCP Theorem: An Introduction (DIMACS)

%\subsection{Examples of PCPs}


%PCPs and CSPs.
%\subsection{``Naive" PCP for NP}

\section{Label-Cover and Projection Games}
%Definition of Label-Cover
We now introduce a problem which manages to provide a natural paradigm for capturing the essence of CSPs and proving inapproximability results. These ``projection games" were introduced by Bellare, Goldreich, and Sudan \cite{bellare1998free}. The $\NP$-hardness of the gap problem version of Label Cover was used by H\aa stad to show tight inapproximability results for $\mathsf{Max3SAT}$ and $\ELin$ \cite{haastad2001some}.

\begin{definition}
A \emph{Label Cover (LC) Problem} instance $\mathcal{G}$ is defined by a bipartite graph $(A \sqcup B,E)$, finite alphabets $\Sigma_A, \Sigma_B$, and a set of projections $\pi_e:\Sigma_A \rightarrow \Sigma_B$ for every edge $e \in E$. Define an \emph{assignment} as consisting of two maps $\mathfrak{A}: A \rightarrow \Sigma_A$, $\mathfrak{B}: B \rightarrow \Sigma_B$. An edge $e = (a,b) \in E$ is said to be satisfied by this assignment if the assignment is compatible with projection $\pi_e$:

\begin{equation}
  \pi_e(\mathfrak{A}(a)) = \mathfrak{B}(b)
\end{equation}

The value of this game will be

\begin{equation} \label{optvalLC}
  \mathsf{Opt}(\mathcal{G}) = \max_{(\mathfrak{A},\mathfrak{B})} \mathbb{E}_{e \sim E}[e \text{ satisfied}]
\end{equation}
In other words, the value will be the largest fraction of edges satisfied by any assignment to the vertices. The corresponding gap problem for Label Cover, $\mathsf{Gap}_{\alpha,\beta}\mathsf{LC}$, is defined as the promise problem:
%
\begin{align*}
    \mathsf{YES} & = \{\mathcal{G} \mid \mathsf{Opt}(\mathcal{G}) \geq \beta\} \\
    \mathsf{NO} & = \{\mathcal{G} \mid \mathsf{Opt}(\mathcal{G}) < \alpha \}
\end{align*}

In the case of perfect completeness, we abbreviate $\mathsf{Gap}_{\alpha,1}\mathsf{LC}$ as simply $\mathsf{Gap}_{\alpha}\mathsf{LC}$.
\end{definition}

There are a few observations worthy of mentioning here. The first regards a type of equivalence between CSP instances and Label Cover instances. Specifically, let $\I$ be an instance of a given CSP $\Psi$ over domain $\Omega$. We can translate this CSP instance into a Label Cover instance as follows: Let the left-hand partition $A$ of our bipartite graph be indexed by the set of constraint tuples $(S,\psi)$ and the right-hand partition $B$ be indexed by the variables of the CSP $V$. Draw an edge from a constraint tuple $(S,\psi)$ to a variable $v$ if that variable appears in $S$. Set $\Sigma_A = \Omega^r, \Sigma_B = \Omega$ where $r$ is the arity of the CSP, and for every edge $e = ((v_1,\cdots,v_r), \psi), v)$ define the projection $\pi_e:\Sigma_A \rightarrow \Sigma_B$ to be

\[ \pi_e(\omega_1, \cdots, \omega_r) =  \omega_i \text{ if } v_i = v\]\newline

%Finish this.

On the other hand, every Label Cover instance can be seen as a $2$CSP over a sufficiently large domain: the predicates of the CSP would be all $2$-ary predicates $\pi:\Sigma_A \times \Sigma_B \rightarrow \{0,1\}$ representing every possible map from $\Sigma_A \rightarrow \Sigma_B$. Thus, the domain of our CSP can be defined as $\Omega = \Sigma_A \cup \Sigma_B$. The corresponding instance of this CSP would be $(S,\pi_e)$ where $\pi_e$ represents the predicate corresponding to the edge $e$'s projection map $\pi_e$ and $S = (a,b)$ would be the vertices of $e$ between $A$ and $B$ respectively.

\begin{theorem} (Weak Projection Games Theorem)
Label Cover is $\NP$-hard to approximate within some constant.
\end{theorem}

%alphabet size.
%examples.
%trivial NP hardness proof.

%\subsection{}
%Arora, Barak
\subsection{Some Structural Results of PCPs}


\subsection{Raz's Parallel Repetition Theorem}

\begin{theorem} (Projection Games Theorem) \label{labelcoverhard}
  For every $\epsilon > 0$, there exist alphabets $\Sigma_A, \Sigma_B$ where $|\Sigma_A|,|\Sigma_B| \leq \mathsf{poly}(\frac{1}{\epsilon})$ such that $\mathsf{Gap}_\epsilon\mathsf{LC}$ is $\NP$-hard.
\end{theorem}

\section{H\aa stad's 3-bit PCP}
%Arora, Barak
\subsection{Inapproximability Results for $\ELin$ and $\mathsf{Max3Sat}$}
The significance of H\aa stad's PCP for $\NP$ is its use in showing tight inapproximability results for $\ELin$ and hence $\mathsf{Max3Sat}$. The theorem is first stated below:

\begin{theorem}(H\aa stad's 3-bit PCP, \cite{haastad2001some})
For every $\delta > 0$ and $L \in \NP$, there exists a PCP verifier for $L$ over the boolean alphabet such that for every
\begin{itemize}
  \item The verifier $V$ queries 3 bits of the proof $x_{q_1},x_{q_2}, x_{q_3} \in \pi$ such that verification predicate is a three variable linear equation over $\mathbb{F}_2$ depending on the queried bits $x_{q_1},x_{q_2}, x_{q_3}$.
  \item If $x \in L$, then there exists a proof $\pi$:
        \begin{equation}
          \mathbb{P}[V(x,\pi) = 1] \geq 1 - \delta
        \end{equation}
  \item If $x \not\in L$, then for all proofs $\pi$:
        \begin{equation}
          \mathbb{P}[V(x,\pi) = 1] \geq \frac{1}{2} + \delta
        \end{equation}
\end{itemize}

\end{theorem}

\subsection{The Long Code}
To be written...

\subsection{Aside on Dictatorship Testing}
The first step towards testing if an input boolean function $f:\{-1,1\}^n \rightarrow \{-1.1\}$ is a dictator arises from the Blum-Luby-Rubinfeld Test (BLR) which we restate below: \newline

\begin{enumerate}
  \item Sample $x,y \sim_R \{-1,1\}^n$
  \item Accept iff $f(x)f(y) = f(xy)$
\end{enumerate}

\begin{theorem}
  Suppose the BLR test accepts $f:\{-1,1\}^n \rightarrow \{-1,1\}$ with probability $1 - \epsilon$, then $f$ is $\epsilon$-close to a linear function $\chi_{S*}$ for some $S^* \subseteq [n]$.
\end{theorem}
%
Certainly the dictators $\chi_i, \; i \in [n]$ are linear functions. Hence, if $f = \chi_i$ for some $i$, then the BLR test accepts with probability one. However, we have the rest of the parity functions $\chi_{S}, |S| \geq 2$ which the BLR Test cannot distinguish. We need to amend the test to ensure that parity functions of higher weight are rejected with high probability. H\aa stad proposed modifiying the vanilla BLR test to add noise to the sampled product $xy$. Although this sacrifices perfect completeness, it penalizes large parity functions: \newline
%
\begin{enumerate}
  \item Sample $x,y \sim \{-1,1\}^n$
  \item Sample $z \sim N_{1-2\epsilon}(xy)$
  \item Accept iff $f(x)f(y)=f(z)$
\end{enumerate}
%

\begin{lemma} (Completeness of the Noisy BLR Test)
 If $f = \chi_{i}$ is a dictator for some $i$, then
  \[ \mathbb{P}[\text{ Noisy BLR test accepts }] \geq 1 - \epsilon \]
\end{lemma}
\begin{proof}
  Note that if $z \sim N_{1-2\epsilon}(xy)$ for some $x \in \{-1,1\}^n$, then $y$ can be expressed as below:
  \begin{equation}
    z_i = \begin{cases}
             \phantom{-} x_iy_i \text{ with probability } 1- \epsilon \\
             - x_iy_i \text{ with probability } \epsilon
          \end{cases}
  \end{equation}
  By the acceptance criterion,
  \[ f(z) = f(x)f(y) \implies z_i = x_iy_i \]
  This occurs with probability $1- \epsilon$ by the observation above, yielding completeness as claimed.
\end{proof}
%
%
\begin{lemma} Suppose that for some constant $\nu > 0$:
\begin{equation*}
  \mathbb{P}[\text{ Noisy BLR test accepts }] \geq \frac{1}{2} + \nu
\end{equation*} then
\begin{equation}
  2\nu \leq \sum_{S \subseteq [n]} \hat{f}(S)^3(1-2\epsilon)^{|S|}
\end{equation}
\end{lemma}
%
\begin{proof}
  The proof begins by noticing the accepting probability can be expressed as:
  \[ \mathbb{P}[\text{ Noisy BLR test accepts }] = \frac{1}{2} + \frac{1}{2}\mathbb{E}_{x,y,z} \left[f(x)f(y)f(z) \right] \]
  By our assumption, we prove that:
  \begin{align*}
    \frac{1}{2} + \nu & \leq \frac{1}{2} + \frac{1}{2}\mathbb{E}_{x,y,z} \left[f(x)f(y)f(z) \right] \implies \\[0.7ex]
    2\nu & \leq \mathbb{E}_{x,y,z} \left[f(x)f(y)f(z) \right] \implies \\[0.7ex]
    2\nu & \leq\mathbb{E}_{x,y,z} \left[f(x)f(y) \mathbb{E}_{z \sim N_{1 - 2\epsilon}(xy)} \left[f(z)\right]\right] \implies \\[0.7ex]
    2\nu & \leq \mathbb{E}_{x}\left[ f(x)\mathbb{E}_{y}\left[ f(y)\mathcal{T}_{1-2\epsilon}f(xy)\right]\right] \\[0.7ex]
    2\nu & \leq \mathbb{E}_{x}\left[ f(x) (f * \mathcal{T}_{1-2\epsilon}f)(x)\right] \implies\\[0.7ex]
    2\nu & \leq \sum_{S \subseteq} \widehat{f}(S)\widehat{f}(S)\widehat{\mathcal{T}_{1-2\epsilon}f}(S) \implies \\[0.7ex]
    2\nu & \leq  \sum_{S \subseteq} \widehat{f}(S)^3 (1-2\epsilon)^{|S|}
  \end{align*}
  as claimed.
\end{proof}
%
\begin{corollary}
  (Soundness of the Noisy BLR Test) There exists some $S^* \subseteq [n]$ such that
  \begin{equation}
    |\widehat{f}(S^*)| \geq 2\nu \quad |S^*| \leq O\left(\frac{1}{\epsilon}\log{\frac{1}{\nu}}\right)
  \end{equation}
\end{corollary}

\section{Unique Games}
\subsection{Definitions}
The PCP Theorem culminated in a proof of the $\NP$-hardness of Label Cover by H\aa stad. Although these results gave proofs of the $\NP$-hardness of $\mathsf{Gap}_{\frac{7}{8} + \epsilon, 1- \epsilon}$-$\mathsf{Max3SAT}$ and $\mathsf{Gap}_{\frac{1}{2} + \epsilon, 1- \epsilon}$-$\mathsf{MaxE3Lin}$, similar hardness proofs for other canonical problems such as $\mathsf{MaxCut}$ didn't seem to follow from these ideas. In his seminal paper, Khot proposed a relaxation of the Label Cover Problem \cite{khot2002power}. The instances of this relaxed version are called \emph{Unique Games}:

\begin{definition}
  A \emph{Unique Label Cover Problem} with $m$ labels (UniqueLC($m$)) instance $\mathcal{U}$ is defined by a bipartite graph $(A \sqcup B,E)$ where $|A| = |B| = n$ for some $n \in \mathbb{N}$, finite alphabet $\Sigma_A = \Sigma_B = \Sigma$ such that $|\Sigma| = m$, and a set of \emph{permutations} $\pi_e:[m] \rightarrow [m]$ for every edge $e \in E$. Define an \emph{assignment} as consisting of a map $\sigma: A \sqcup B \rightarrow [m]$. An edge $e = (a,b) \in E$ is said to be satisfied by this assignment if the assignment is compatible with projection $\pi_e$:

  \begin{equation}
    \pi_e(\sigma(a)) = \sigma(b)
  \end{equation}

  The value of this game will be

  \begin{equation} \label{optvalLC}
    \mathsf{Opt}(\mathcal{G}) = \max_{\sigma} \mathbb{E}_{e \sim E}[e \text{ satisfied}]
  \end{equation}
  In other words, the value will be the largest fraction of edges satisfied by any assignment to the vertices. The corresponding gap problem for Label Cover, $\mathsf{Gap}_{\alpha,\beta}\mathsf{UniqueLC}(m)$, is defined as the promise problem:
  %
  \begin{align*}
      \mathsf{YES} & = \{\mathcal{U} \mid \mathsf{Opt}(\mathcal{U}) \geq \beta\} \\
      \mathsf{NO} & = \{\mathcal{U} \mid \mathsf{Opt}(\mathcal{U}) < \alpha \}
  \end{align*}

  In the case of perfect completeness, we abbreviate $\mathsf{Gap}_{\alpha,1}\mathsf{UniqueLC}(m)$ as simply $\mathsf{Gap}_{\alpha}\mathsf{UniqueLC}(m)$.
\end{definition}

In addition, Khot formulated the \emph{Unique Games Conjecture} and utilized it to prove several inapproxability results assuming the conjecture is true.

\begin{conjecture} (Unique Games Conjecture \cite{khot2002power})
  For any constant $\delta > 0$, there exists sufficiently large $m \in \mathbb{N}$ such that $\mathsf{Gap}_{\delta,1-\delta}\mathsf{UniqueLC}(m)$ is $\NP$-hard.
\end{conjecture}

\begin{remark}
  The $1-\delta$ constant is crucial to the validity of the conjecture. For an instance of $\mathsf{UniqueLC}(m)$ for any $m$ with a \emph{guaranteed} solution, there exists a polynomial-time algorithm finding an assignment which satisfies all projection constraints: First, start with a vertex and set it to a label. If the vertex is an endpoint of an edge $e$, follow $e$ to the other side and find a label which satisfies as many neighbors as possible. Repeat in a breadth-first search fashion. In virtue of the projection maps being permutations, this amounts to searching for the guaranteed solution in time $O(mn^2)$.
\end{remark}

\begin{example}
  The $\mathsf{MaxCut}$ problem for an input graph $G = (V,E)$ can be cast as a $\mathsf{UniqueLC}(|V|)$ instance. The two partitions of the bipartite graph $A,B$ will be indexed by the vertices $V$. Draw an edge between two vertices $v_1,v_2$ if $(v_1,v_2) \in E$. Set the alphabet to be $\Sigma =\{-1,1\}$ and the projection maps to be the ``swap" map $-1 \mapsto 1, 1 \mapsto -1$.
\end{example}

\begin{example}
  $\mathsf{MaxE2LinModp}$ for prime $p$ denotes the problem of finding an assignment which maximizes the number of satisfied linear constraints consisting of exactly two variables over field $\mathbb{F}_p$. An example of an instance of this problem over variables $x_1,x_2,x_3,x_4$ is shown below:

  \begin{equation*}
    \begin{alignedat}{3}
      x_1 & +{} & x_3 & = 3 \\
      x_2 & +{}  & x_4 & = 2 \\
      x_1 & +{} & x_4 & = 1
    \end{alignedat}
  \end{equation*}

  An instance of this problem can also be translated as an instance of $\mathsf{UniqueLC}(p)$.
\end{example}

\begin{definition}
  A promise problem $P$ is said to be \emph{$\UG$-hard} if there is some constant $\delta > 0$ such that for all $m \in \mathbb{N}$, there exists a polynomial-time reduction $f$ from $\GapDeltaULC$ to $P$, in the sense that for a $\GapDeltaULC$ instance $\mathcal{U}$:
  \begin{align*}
    \mathsf{Opt}(\mathcal{U})  \geq 1 - \delta & \implies f(\mathcal{U}) \in \mathsf{Yes} \\
    \mathsf{Opt}(\mathcal{U})  < \delta & \implies f(\mathcal{U}) \in \mathsf{No}
  \end{align*}
\end{definition}

%\subsection{Is the Unique Games Conjecture True?}

\section{UG-hardness of MAXCUT}
\subsection{Intuitions}
Recall that the keen insight behind H\aa stad's 3-bit PCP for $\NP$ is the embedding of a dictatorship test within a Label Cover instance. In a similar spirit, the proof for the optimality of the Goemans-Williamson algorithm will hinge on crafting a clever dictatorship test embedded within a $\MaxCut$ instance. By composing this test with

\subsection{Goemans-Williamson Algorithm for $\MaxCut$} \label{goemans}
 Example \ref{maxcutexample} presented an LP-based approximation algorithm for $\MaxCut$. Goemans and Williamson \cite{goemans1995improved} designed an SDP to give an $\alpha_{GW}$-approximation algorithm for $\MaxCut$ where:

\begin{equation}
  \alpha_{GW} = \min_{-1 \leq \rho \leq 1} \frac{2}{\pi}\frac{\cos^{-1}(\rho)}{1 - \rho} \approx 0.87856
\end{equation}

To begin, a semi-definite program is a generalization of a linear program where instead of optimizing over a vector of variables $\vec{x}$, the program considers a positive semi-definite matrix of variables i.e a matrix whose eigenvalues are non-negative. An equivalent formulation considers inner products between pairs of vectors:

\begin{align*}
     \max & \sum_{i,j} c_{ij}\langle v_i, v_j \rangle \\
     \text{ under constraints }& a_{ij}^k\langle v_i, v_j \rangle \leq b_{ij}^k \\
     v_1, \cdots v_n \in \mathbb{R}, \quad & k \in [C] \text{ constraints }
\end{align*}

Note that this form also subsumes quadratic programming by setting $c_{ij} = a_{ij}^k = b_{ij}^k = 0$ for all $i \neq j$ and $k \in [C]$. The Goemans-Williamson algorithm concerns the solution to the semi-definite program given some graph $G = (V,E)$: \newline
%
\begin{align}
  & \max \sum_{i,j} \frac{1 - \langle v_i, v_j \rangle}{2} \\
  & \langle v_i, v_i \rangle = 1 \text{ for all } i \in V
\end{align}

Let us first show that indeed the program is a relaxation of the integer program crafted in Example \ref{maxcutexample}. Indeed, if we set any unit vector $\vec{u}$ such that for a cut defined by $(S,\bar{S})$:
%
\begin{align*}
  \vec{v_i} = \begin{cases}
                u \text{ if } i \in S \\
                -u \text{ if } i \not\in S
              \end{cases}
\end{align*}

A direct calculation yields that the term $\frac{1 - \langle v_i, v_j \rangle}{2} = 0$ if $v_i = v_j = u$ else it is equal to 1. Thus, by applying this observation to the maximum cut,
%
 \[\mathsf{Opt}_{GW}(G) \geq \MaxCut(G)\]

So far the semi-definite program seems to be rather simple. The insight made by Goemans and Williamson lies in the rounding procedure of the solution outputted by the program. The procedure proceeds by first drawing a random hyperplane passing through the origin and taking the two partitions of the cut $S_+, S_-$ to be the solution $v_i$ which lie on positive and negative sides of the hyperplane respectively. We can calculate the expected size of the cut:
%
\begin{align*}
\mathbb{E}[|E(S_+,S_-)|] = \sum_{(i,j) \in E} \mathbb{P}[v_i,v_j \text{ lie on different sides of the hyperplane}]
\end{align*}

Now if $\theta_{ij}$ denotes the angle between $v_i,v_j$, a simple geometric argument shows that:

\[\mathbb{P}[v_i,v_j \text{ lie on different sides of the hyperplane}] = \frac{\theta_{ij}}{\pi} \]

By definition of the dot product:

\[\frac{1 - \langle v_i, v_j \rangle}{2} =  \frac{1 - \cos(\theta_{ij})}{2} \]

and the magical inequality:

\[ \frac{\theta_{ij}}{\pi} \geq \alpha_{GW} \cdot \frac{1 - \cos(\theta_{ij})}{2}\]

the two can be combined to finally yield that:

\begin{equation}
  \frac{\mathbb{E}[|E(S_+,S_-)|]}{\mathsf{Opt}_{GW}(f)} \approx \alpha_{GW}
\end{equation}

\subsection{$\MaxCut$ is $\UG$-hard}
After introducing the basic background, we are finally ready to show a non-trivial inapproximability result assuming the UGC:

\begin{theorem} \label{maxcutughard} ($\MaxCut$ is $\UG$-hard) The problem $\GapMaxCut$ is $\UG$-hard
\end{theorem}
Note that an immediate corollary of Theorem and the UGC is that the Goemans-Williamson algorithm is tight:

\begin{corollary}
Assuming the UGC, if an $\alpha_{GW}$-approximation algorithm for $\MaxCut$ exists, then $\P = \NP$.
\end{corollary}
%
\begin{proof}
Using Theorem \ref{GapCSPtoAlgHard}, we see that for small $\epsilon$:
\[ \frac{\GWConstant + \epsilon}{\frac{1-\rho}{2} - \epsilon} \approx \frac{2}{\pi}\frac{\cos^{-1}(\rho)}{1 - \rho} \]
Setting $\rho \approx -0.6934$ yields the desired result.
\end{proof}

Onwards to the proof of Theorem \ref{maxcutughard}. We first reason about the relationship between $\ULC$ instances and $\MaxCut$ instances. In particular, we wish to demonstrate that proving $\UG$-hardness for $\MaxCut$ is equivalent to constructing a $2$-query PCP for $\GapULC$ for all $m$.

\begin{theorem}
  $\mathsf{Gap}_{\alpha,\beta}\mathsf{MaxCut}$ is $\UG$-hard iff there exists a constant $\delta > 0$ such that for all $m$ there exists a 2-query PCP for $\mathsf{Gap}_{\delta, 1-\delta}\ULC$ with completeness $\beta$ and soundness $\alpha$.
\end{theorem}
%
\begin{proof}
  First assume that for some $\delta > 0$ and all $m$, there exists a polynomial-time reduction from $\GapDeltaULC$ to $\mathsf{Gap}_{\alpha,\beta}\mathsf{MaxCut}$. Let $G=(V,E), \mathcal{C}$ be the graph outputted by reduction and the cut outputted by the cut approximation algorithm. Construct a PCP samples two vertices from $v_1, v_2 \sim V$ by querying the proof tape encoding $G,\mathcal{C}$ and outputs ``accept" if $(v_1,v_2) \in \mathcal{C}$. The completeness and soundness parameters immediately follow from definition of $\mathsf{Gap}_{\alpha,\beta}\mathsf{MaxCut}$. Conversely, assume the existance of a 2-query PCP for $\GapDeltaULC$ with the above properties. We will calculate the acceptance probabilities of this PCP given a proof string and an input instance by finding the max cut. Let the vertices be the proof locations of the input proof string $\pi$. It suffcies to draw an edge between two vertices weighted by the probability their corresponding proof locations are queried by the verifier.
\end{proof}

Thus, we can proceed by constructing a 2-query PCP for $\GapDeltaULC$ with the completeness $\frac{1-\rho}{2} - \epsilon$ and soundness $\GWConstant + \epsilon$. As with H\aa stad's 3-bit PCP, we aim to construct a dictatorship test which generates the parameters needed. Once again, a proof string $\pi$ containing an assignment of labels to the vertices of a $\ULC$ instance would be encoded into a truth table of a dictator i.e into a \emph{long code}. \newline

%
%
%
\subsection{Majority is the Stablest (MIS)}
%
Before we craft our dictator test, we require a tool bounding the noise sensitivity of a boolean function with small low-degree influence. This is captured by the ``Majority is the Stablest" theorem:

\begin{theorem} (``Majority is the Stablest" (MIS) \cite{mossel2005noise})
For every $\epsilon > 0$, $\rho \in (-1,0)$, there exists $\tau > 0$ such that if for all $i \in [n]$, $\mathsf{Inf}_i(f) < \tau$ for function $f:\{-1,1\}^n \rightarrow [-1,1]$, then

\begin{equation}
  \mathsf{NS}_{\rho}(f) < \GWConstant + \epsilon
\end{equation}
\end{theorem}

Recall that the noise sensitivity of a boolean function $f:\{-1,1\}^n \rightarrow \{-1,1\}$ is defined as:

\begin{equation} \label{NSdef}
  \mathsf{NS}_\rho(f)  = \mathbb{P}_{y \sim N_{\rho}(x), x} [f(x) \neq f(y)]
\end{equation}

where $y \sim N_\rho(x)$ refers to sampling a string $y$ under the procedure:

\begin{equation*}
  y_i = \begin{cases}
        x_i \quad \text{with probabilty} \frac{1 + \rho}{2} \\
        -x_i \quad \text{with probabilty} \frac{1 - \rho}{2}
        \end{cases}
\end{equation*}
\newline

Now equation \ref{NSdef} can be re-expressed in terms of the noise stability of $f$:

\begin{equation}
  \mathsf{NS}_\rho(f) = \frac{1}{2} - \frac{1}{2}\mathsf{Stab}_\rho(f)
\end{equation}
where

\begin{equation*}
  \mathsf{Stab}_\rho(f) = \mathbb{E}_{y \sim N_{\rho}(x), x}[f(x)f(y)]
\end{equation*}
Through Fourier-analytic techniques, we derive that:

\begin{equation}
  \mathsf{NS}_\rho(f) = \frac{1}{2} - \frac{1}{2}\sum_{S\subseteq [n]}\hat{f}(S)^2\rho^{|S|}
\end{equation}

A generalization of the MIS theorem will serve our purposes:

\begin{theorem} \label{generalMIS} (Generalized MIS \cite{khot2007optimal},\cite{mossel2005noise})
  For all $\epsilon > 0$, $\rho \in (0,1)$, there exists some $\tau > 0$ and finite $d$ such that if $f:\{-1,1\}^n \rightarrow [-1,1]$ and for all $i \in [n]$, $\mathsf{Inf}^{\leq d}_i(f) \leq \tau$:
  \begin{equation}
    \frac{1}{2} - \frac{1}{2}\sum_{S\subseteq [n]}\hat{f}(S)^2\rho^{|S|} < \GWConstant + \epsilon
  \end{equation}
\end{theorem}

%
%
%
\subsection{The 2-query PCP}
With the MIS theorem, we can begin our analysis of the promised 2-query PCP. Before the procedure, define $(x \circ \pi)_i = x_{\pi(i)}$ for $x \in \{-1,1\}^n, \; i \in [n], \; \pi \in S_n$.  \newline

\begin{enumerate}
  \item Sample a vertex $a \in A$ uniformly.
  \itemsep1em
  \item Sample two of its neighbors $b,b' \in B$ uniformly. Let $\pi_{b,a}$ and $\pi_{b',a}$ denote the \emph{inverses} of the constraints associated to $(a,b), (a,b')$ respectively.
  \item Sample $x \sim \{-1,1\}^m$ and $y \sim N_\rho(x)$
  \item Accept if $f_b(x \circ \pi_{b,a}) \neq f_b'(y \circ \pi_{b',a})$
\end{enumerate}

For the proof of Theorem \ref{maxcutughard}, invoke Theorem \ref{generalMIS} for $\frac{\epsilon}{2}$ and $\rho$ assumed. This will yield a $\tau$ and a degree upper bound $d$. Set parameter $$ \delta = \frac{\epsilon \tau^2}{8d} $$


\subsubsection{Completeness}
%
Suppose we have a $\ULC$ instance $\mathcal{U}$ such that $\mathsf{Opt}(\mathcal{U}) \geq 1 - \delta$. Through a simple union bound argument, the probability that both $\pi_{b,a},\pi_{b',a}$ are satisfied by the assignment is at least $1- 2\delta$. Now if both are indeed satisfied and the test accepts, it must be true that:
%
\begin{align*}
    f_b(x \circ \pi_{b,a}) \neq f_b'(y \circ \pi_{b',a}) & \iff
    (x \circ \pi_{b,a})_{\sigma(b)} \neq (x \circ \pi_{b',a})_{\sigma(b')} \\
    & \iff x_ {\pi_{b,a}(\sigma(b))} \neq x_ {\pi_{b',a}(\sigma(b'))} \\
    & \iff x_{\sigma(a)} \neq y_{\sigma(a)}
\end{align*}
where for the last equivalence, we invoked the assumption that $\sigma$ satisfies both $\pi_{b,a},\pi_{b',a}$. Recall that $\sigma$ is the assignment of labels to the vertices of the biparitite graph. The last expression occurs with probabilty $\frac{1 - \rho}{2}$ by step three of the verification algorithm. Hence,
\[ \mathbb{P}[\text{Test accepts}] \geq \frac{(1-2\delta)(1 - \rho)}{2}\]

By observing that $\delta < \frac{\epsilon}{2}$, the inequality above reduces to:

\begin{equation}
  \mathbb{P}[\text{Test accepts}] \geq \frac{(1-\epsilon)(1 - \rho)}{2} \geq \frac{1 - \rho}{2} - \epsilon
\end{equation}
as desired.

\subsubsection{Soundness}
To show soundness, we prove the contrapositive, namely if
$$ \mathbb{P}[\text{Test accepts}] \geq \frac{\cos^{-1}(\rho)}{\pi} + \epsilon $$
then there exists a labeling which satisfies more than a $\delta$ fraction of constraints.
%
\begin{proof}
First:
\begin{align} \label{step1}
  \mathbb{P}[\text{Test accepts}] = \mathbb{E}_{a,b,b'}\left[
  \mathbb{E}_{x,y\sim N_{\rho}(x)}\left[
  \frac{1}{2} - \frac{1}{2}f_b(x \circ \pi_{b,a})f_{b'}(y \circ \pi_{b',a})\right]\right] \geq \frac{\cos^{-1}(\rho)}{\pi} + \epsilon
\end{align}
An averaging argument on the vertex $a \in A$ tells us that, since the test passes with at least $\GWConstant + \epsilon$ probability, there must exist at least $\epsilon/2$ fraction of the vertices in $A$ such that the test conditioned on picking $a$ from this fraction passes with probability at least $\GWConstant + \epsilon/2$. Otherwise, if there existed fewer than a $\epsilon/2$ fraction such that the test passed with at least this probability, then the total probability of the test passing is \emph{at most} $(\epsilon/2)\cdot 1 + (1 - \epsilon/2)(\GWConstant + \epsilon/2) < \GWConstant + \epsilon$, which is a contradiction. Let us label the vertices picked from this $\epsilon/2$ fraction as \emph{good} vertices. \newline

So say we picked one of these good vertices say $a$. Let us define the below function:

\begin{definition}
  Define $g_a: \{-1,1\}^m \rightarrow [-1,1]$ as
  \begin{equation}
    g_a(x) = \mathbb{E}_{b}\left[f_b(x \circ \pi_{b,a}) \right]
  \end{equation}
  where the expectation is drawn uniformly over the neighbors $b$ of $a$.
\end{definition}
%
This allows us to re-express the inequality \ref{step1}:

\begin{align}
  & \mathbb{E}_{b,b'}\left[
  \mathbb{E}_{x,y\sim N_{\rho}(x)}\left[
  \frac{1}{2} - \frac{1}{2}f_b(x \circ \pi_{b,a})f_{b'}(y \circ \pi_{b',a})\right]\right] \\
  =  & \mathbb{E}_{x,y\sim N_{\rho}(x)}\left[
    \frac{1}{2} - \frac{1}{2}\mathbb{E}_{b}\left[f_b(x \circ \pi_{b,a})\right]\mathbb{E}_{b'}\left[f_{b'}(y \circ \pi_{b',a})\right]\right] \\
     = &  \frac{1}{2} - \frac{1}{2}\mathbb{E}_{x,y\sim N_{\rho}(x)}\left[g_a(x)g_a(y)\right]\\
     = & \frac{1}{2} - \frac{1}{2}\mathsf{Stab}_{\rho}(g_a) \\
   \geq & \frac{\cos^{-1}(\rho)}{\pi} + \epsilon/2
\end{align}
By invoking the contrapositive of the generalized MIS theorem (Theorem \ref{generalMIS}) on $g_a$, we deduce that there must exist some index $i_a \in [m]$ such that:
%
\begin{equation} \label{contra}
\mathsf{Inf}_{i_a}^{\leq d}(g_a) \geq \tau
\end{equation}
%
Fortunately, there is a method to equate the Fourier coefficients $g_a$ with those of $f_b$:

\begin{align*}
  g_a(x) & = \mathbb{E}_{b}\left[ f_b(x \circ \pi_{b,a})\right] \\
         & = \mathbb{E}_{b}\left[ \sum_{S \subseteq [n]} \hat{f_b}(S)\chi_S(x \circ \pi_{b,a})\right] \\
         & = \mathbb{E}_{b}\left[\sum_{S \subseteq [n]} \hat{f_b}(S)\chi_{\pi_{b,a}(S)}(x) \right] \\
         & =  \mathbb{E}_{b}\left[\sum_{S \subseteq [n]} \hat{f_b}(S)\chi_{\pi_{b,a}(S)}(x) \right] \\
         & = \sum_{S \subseteq [n]}\mathbb{E}_{b}\left[ \hat{f_b}(\pi_{b,a}^{-1}(S))\right]\chi_{S}(x)
\end{align*}
where the last equality follows from reparameterizing the sum. By expanding $g_a$ on the left-hand side of the equality into its Fourier expansion, from the orthogonality of characters:
%
\begin{equation}
  \hat{g_a}(S) = \mathbb{E}_{b}\left[ \hat{f_b}(\pi_{b,a}^{-1}(S))\right]
\end{equation}
Starting from inequality \ref{contra},
%
\begin{align*}
  \tau \leq \mathsf{Inf}_{i_a}^{\leq d}(g_a)
  & = \sum_{\overset{S \subseteq [n], i_a \in S}{|S| \leq d}}  \hat{g_a}^2(S) \\
  & = \sum_{\overset{S \subseteq [n], i_a \in S}{|S| \leq d}}  \left(\mathbb{E}_{b}\left[ \hat{f_b}(\pi_{b,a}^{-1}(S))\right] \right)^2 \\
 & \leq \sum_{\overset{S \subseteq [n], i_a \in S}{|S| \leq d}}  \mathbb{E}_{b}\left[ \hat{f_b}^2(\pi_{b,a}^{-1}(S))\right] \\
 & \mathbb{E}_{b}\left[ \sum_{\overset{S \subseteq [n], i_a \in   S}{|S| \leq d}}  \hat{f_b}^2(\pi_{b,a}^{-1}(S))\right] \\
 & = \mathbb{E}_{b}\left[ \mathsf{Inf}^{\leq d}_{\pi_{b,a}^{-1}(i_a)}(f_b) \right]
\end{align*}
The inequality uses Cauchy-Schwarz.
We use another averaging argument to see that there must exist at least a $\tau/2$ fraction of $a$'s neighbors $b$ such that $$\mathsf{Inf}^{\leq d}_{\pi_{b,a}^{-1}(i_a)}(f_b) \geq \tau/2$$
otherwise the total influence term would be at most $\tau/2 \cdot 1 + (1 -\tau/2)(\tau/2) < \tau$.
For each $b$ in that $\tau/2$ fraction, we pick a label uniformly at random from the set:

\[ S_b = \{\ell \mid \mathsf{Inf}^{\leq d}_{\ell}(f_b) \geq \tau/2 \} \]
which must be \emph{non-empty} by the averaging argument made above. Notice that the label picked will satisfy $\pi_{b,a}$ by construction. We can thus lower bound the probability of constraint $\pi_{b,a}$ being satisfied:
%
\begin{equation} \label{probbound}
\mathbb{P}[\pi_{(b,a)} \text{ is satisfied}] \geq \frac{\epsilon}{2}\frac{\tau}{2}\frac{1}{|S_b|}
\end{equation}

Through a Fourier-analytic argument, we can upper-bound $|S_b|$:
\begin{align*}
  \frac{|S_b|\tau}{2} \leq \sum_{i = 1}^{|S_b|} \mathsf{Inf}_i^{\leq d}(f_b) \leq \sum_{i = 1}^m \mathsf{Inf}_i^{\leq d}(f_b) = \sum_{\overset{S \subseteq [m]}{|S| \leq d}} |S| \hat{f}^2(S) \leq d \sum_{S \subseteq [m]}  \hat{f}^2(S) = d
\end{align*}
This yields that $|S_b| \leq \frac{2d}{\tau}$. So the probability bound of \ref{probbound} would become:
%
\begin{equation}
  \mathbb{P}[\pi_{(b,a)} \text{ is satisfied}] \geq \frac{\epsilon}{2}\frac{\tau}{2}\frac{\tau}{2d} = \frac{\epsilon\tau^2}{8d} = \delta
\end{equation}
as desired. This completes the proof of Theorem \ref{maxcutughard}.

\end{proof}


%add a big picture composing both parts.




%\section{Semi-definite Programming and Integrality Gaps}
%\section{Quantum Information Theory}
%\section{Other Connections}

\bibliographystyle{amsplain}
\bibliography{biblio}

\end{document}
