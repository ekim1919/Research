\documentclass{amsart}
\usepackage{complexity}
\setlength{\parindent}{0pt}

\setlength{\textwidth}{\paperwidth}
\addtolength{\textwidth}{-2in}
\calclayout

\newtheorem{theorem}{Theorem}[section]
\newtheorem{corollary}{Corollary}[theorem]
\newtheorem{lemma}[theorem]{Lemma}

\theoremstyle{definition}
\newtheorem{definition}[theorem]{Definition}
\newtheorem{example}[theorem]{Example}
\newtheorem{proposition}[theorem]{Proposition}
\newtheorem{conjecture}[theorem]{Conjecture}
\newtheorem{xca}[theorem]{Exercise}

\newcommand{\I}{\mathcal{I}}
\newcommand{\MaxCNF}{\mathsf{Max3CNF}}
\newcommand{\ELin}{\mathsf{MaxE3Lin}}
\newcommand{\MaxCSP}{\mathsf{MaxCSP}(\Psi)}
\newcommand{\MaxCut}{\mathsf{MaxCut}}
\newcommand{\GapMaxCSP}[2]{\mathsf{Gap_{#1,#2}}\MaxCSP}
\newcommand{\GWConstant}{\frac{\cos^{-1}(\rho)}{\pi}}
\newcommand{\UG}{\mathsf{UG}}
\newcommand{\GapDeltaULC}{\mathsf{Gap}_{\delta, 1-\delta}\ULC}
\newcommand{\GapMaxCut}{\mathsf{Gap}_{\GWConstant + \epsilon, \frac{1-\rho}{2} - \epsilon}\mathsf{MaxCut}}
\newcommand{\ULC}{\mathsf{UniqueLC}(m)}
\newcommand{\GapULC}{\mathsf{GapUniqueLC}(m)}
\newcommand{\LC}{\mathsf{Gap}_\epsilon\mathsf{LC}}


\theoremstyle{remark}
\newtheorem*{remark}{Remark}
\newtheorem*{note}{Note}
\numberwithin{equation}{section}

\theoremstyle{remark}
\newtheorem{notation}[theorem]{Notation}
\begin{document}

\renewcommand{\phi}{\varphi}
%\newcommand{\Max3CNF}{\mathsf{Max3CNF}}

\title{Notes on PCPs and Unique Games}

\author{Edward Kim}
\email{ehkim@cs.unc.edu}

\begin{abstract}
We expound on the basic theory of approximating $\NP$-hard problems. Starting with the PCP Theorem, we introduce background for the Unique Games Conjecture (UGC) and the optimality of the Goemans-Williamson Algorithm for MAXCUT assuming the conjecture is true.
%Finally, we make some connections to Quantum Information Theory, especially Quantum PCPs (qPCPs).
\end{abstract}

\maketitle
\tableofcontents

\noindent These notes borrow heavily from Arora and Barak's textbook \cite{arora2009computational}, O'Donnell's textbook \cite{o2014analysis}, Prahladh Harsha's lecture notes on PCPs and Unique Games given at DIMACS and TIFR\cite{harsha2010limitsTIFR}, \cite{harsha2010limitsDIMACS}, and Venkatesan Guruswami and Ryan O'Donnell's Course at UWash \cite{uwash2005}
. Supplementary reading can be found in Luca Trevisan's exposition on the UGC \cite{trevisan2012khot} and Subhash Khot's survey \cite{khot2005unique}.

\section{Approximation Algorithms and CSPs}
%3.2 Limits of Approximation – Introduction (TIFR)
%AB - Chapter 11
%Explain intuition.
\subsection{Definitions and Examples}
The theory of approximating $\NP$-hard problems roots itself in the following question: ``Is it possible to efficiently approximate $\NP$-complete problem to some arbitrary degree of accuracy?" Since the Cook-Levin Theorem demonstrated that the $\SAT$ decision problem is $\NP$-complete, the question could be rephrased as ``Can we find an efficient algorithhm for $\SAT$ which, given an input formula, outputs an assignment which satisfies a $1-\delta$ fraction of the clauses for any constant $\delta >  0$?" Towards this end, let us first introduce a few motivating examples which will find utility in our forthcoming analysis. \newline

Consider $\mathsf{Max3CNF}$, the problem of, given a 3CNF formula $\phi$ as input, outputting an assignment which maximizes the number of clauses satisfied in $\phi$. For any assignment $x$ to the variables, say as an $n$-bit input string while viewing $\phi:\{0,1\}^n \rightarrow \{0,1\}$ as a boolean function, there is some $0 \leq \rho_{\phi,x} \leq 1$ representing the fraction of clauses satisfied in $\phi$. Take $\mathsf{Opt}(\phi)$ to be the maximum value over all such assignments:

\[ \mathsf{Opt}(\phi) = \max_{x \in \{0,1\}^n}\rho_{\phi,x} \]

A polynomial-time algorithm which solves $\MaxCNF$ is one which takes a 3CNF $\phi$ as input and outputs $\Opt(\phi)$. Naturally, $\mathsf{Max3CNF}$ is $\NP$-hard since its corresponding decision problem $\mathsf{3CNF}$ is $\NP$-complete ($\phi$ is satisfiable iff $\mathsf{Opt}(\phi) = 1$).
%
However, we can ask if there exists an algorithm $A$ which outputs an assignment which satisfies at least some $\beta\cdot\mathsf{Opt}(\phi)$ fraction of the clauses of $\phi$ for some $\beta \leq 1$. To formalize this: \newline
%
\begin{definition} \label{Opt3SATDef}
For $\beta \leq 1$, a polynomial-time algorithm $A$ is deemed as a $\beta$-approximation algorithm for $\mathsf{Max3CNF}$ if $A(\phi)$ outputs an assignment which satisifies at least $\beta\cdot\mathsf{Opt}(\phi)$ fraction of $\phi$'s clauses for every 3CNF instance $\phi$. More specifically, the algorithm $A$ outputs some value such that:
\begin{equation}
  \beta\cdot\mathsf{Opt}(\phi) \leq A(\phi) \leq \mathsf{Opt}(\phi)
\end{equation}
If $\mathsf{Opt}(\phi) = 1$, then $\phi$ is said to be \emph{satisfiable}.
\end{definition}

A canonical example of a simple approximation algorithm for $\mathsf{Max3CNF}$ is the following scheme: for every variable in $\phi$ choose with uniform probability an assignment from $\{0,1\}$. The probability of any given clause being satisfied by such a random assignment is $\frac{7}{8}$. Thus
%
\[ \mathbb{E}_{x \in \{0,1\}^n}[\text{\# satisfied clauses of $\phi(x)$}] = \frac{7m}{8} \]
%
where $m$ is the number of satisfiable clauses of $\phi$, showing that there must exist an assignment which satisfies at least $\frac{7}{8}$ of $\phi$'s clauses. We can derandomize this algorithm to output such an assignment. This gives us a $\frac{7}{8}$-approximation algorithm for $\mathsf{Max3CNF}$. Here's another example:
%
\begin{example} \label{e3linexample}
  Let $\mathsf{MaxE3Lin}$ define the problem of the following form: let $\xi$ be defined as a system of linear equations over the field $\mathbb{F}_2$ where every equation contains \emph{exactly} $3$ variables from a set of $n$ variables $x_1,\cdots,x_n$. Find an assignment to the variables which maximizes the number of satisified linear equations in $\xi$. An example is the following system over variables $x_1,x_2,x_3,x_4$:
  %
  \begin{equation*}
    \begin{alignedat}{3}
      x_1 & +{} &  x_2 & +{} & x_3 & = 0 \\
      x_1 & +{} &  x_2 & +{} &  x_4 & = 1 \\
      x_1 & +{} & x_3 & +{} & x_4 & = 1
\end{alignedat}
\end{equation*}
\end{example}
%
The $\mathsf{E3Lin}$ expression represents a linear system of $\mathsf{E}$xactly $\mathsf{3}$ variables. We can extend the notation treated above for $\mathsf{Max3CNF}$ to this situation. In other words, if $\rho_{\xi,x}$ is the fraction of satisfied linear constraints of $\xi$ under assignment $x \in \mathbb{F}_2^n$.
%
\[\mathsf{Opt}(\xi) =\max_{x \in \{0,1\}^n}\rho_{\xi,x}\]
%
A comment regarding this problem: if the given $\mathsf{E3Lin}$ instance has a guaranteed solution i.e $\mathsf{Opt}(\phi) = 1$, then Gaussian Elimination will always output an assignment which satisfies all linear constaints in polynomial-time. It is more interesting to consider instances where no such solution satisfing all of the constraints exists over $\mathbb{F}_2^n$. In these cases, $\mathsf{Opt}(\xi) = 1 - \epsilon$ for some $0 < \epsilon \leq 1$. This problem also has a fairly simple $\frac{1}{2}$-approximation algorithm: set all $x_1 = \cdots = x_n  = 0$ or $x_1 = \cdots = x_n  = 1$ depending on which satisfies the most constraints. This scheme must satisfy at least $\frac{1}{2}$ of the constraints for any instance $\xi$. \newline

\begin{example} \label{maxcutexample}
The problem $\MaxCut$ will be casted as follows: given an undirected graph $G=(V,E)$, find the largest cut of $G$. There is a straightforward LP formulation for this problem:

\begin{align}
  &  \max \sum_{(u,v) \in E} e_{u,v} \\
  & e_{u,v} \leq
  \min \begin{cases}
      x_1 + x_2 \\
      2 - (x_1 + x_2)
  \end{cases} \\
  & e_{u,v} \in [0,1] \\
  & x_v \in [0,1]
\end{align}
%
where the $e_{u,v}$ variables represent the edges $e \in E$ of the input graph and the $x_v$ represent the vertex variables $v \in V$. This will actually be a relaxation from its respective integer program in which the variables can take the integer values:

\begin{align}
  e_{u,v} = \begin{cases}
              1 \text{ if } e_{u,v} \in \mathcal{C} \\
              0 \text{ otherwise }
            \end{cases}
  x_v = \begin{cases}
              0 \text{ if } v \text{ is in partition } S \\
              1 \text{ otherwise }
        \end{cases}
\end{align}

where $\mathcal{C} \subseteq E$ is the subset of edges constituting the cut and $(S,\bar{S})$ is the partition defining the cut. Naturally, a solution to the integer program will also be a solution to the LP above, which implies that

\[ \mathsf{Opt}_{LP} \geq \MaxCut(G)\]

There actually exists an Semi-definite Programming (SDP) relaxation for $\MaxCut$ called the \emph{Goemans-Williamson algorithm} which will be treated in Section \ref{goemans}.

\end{example}

\subsection{Constraint Satisfaction Problems}

 Generally speaking, it's useful to concretize these problems into an unified framework. The examples presented in the last section have some common interpretation shared between them: they could all be seen as manifestations of Constaint Satisfaction Problems (CSP):
%
\begin{definition}
  Let $\Omega$ be a finite set deemed as the \emph{domain}. A constraint satisfaction problem (CSP) $\Psi$ over domain $\Omega$ is a finite set of predicates $\psi:\Omega^r \rightarrow \{0,1\}$ where $r$ is the \emph{arity} of predicate $\psi$. The predicates can be of different arities. The \emph{arity} of the CSP $\Psi$ would be the maximum of all the arities of the predicates in $\Psi$. \newline

  \noindent An \emph{instance} $\mathcal{I}$ over CSP $\Psi$ is a set of tuples $(S,\psi)$ where if $r$ is the arity of predicate $\psi$, $S = (v_1,\cdots,v_r)$ is some ordered tuple of variables taken from finite set $V$ consisting of variables contained in CSP $\Psi$. These tuples are called the \emph{constraints} of $\mathcal{I}$. In addition, we add the condition that every variable show up in at least one constraint. Variables which do not appear in any of the constraints can simply be removed. \newline

  \noindent Now given some instance $\mathcal{I}$, an \emph{assignment} $F:V \rightarrow \Omega$ is simply some  map between the variables and the domain. We say that $F$ satisfies a constaint $(S,\psi)$ if $\psi(F(S)) = 1$. For tuple $S = (v_1, \cdots, v_r)$, we define $F(S) = (F(v_1), \cdots, F(v_r))$. Consider the fraction of constraints of satisfied by $F$ in $\mathcal{I}$:
  %
  \begin{equation}
  \mathsf{Val}_{\I}(F) = \mathbb{E}_{(S, \psi) \sim \I}[\psi(F(S))]
  \end{equation}
  By taking the maximum fraction over all such assignments:
  %
  \begin{equation}
  \mathsf{Opt}(\I) = \max_{F: V \rightarrow \Omega} \mathsf{Val}_{\I}(F)
  \end{equation}
\end{definition}

  %
  \begin{example} \hfill{}
    \begin{itemize}
      \item For $\MaxCNF$, the domain would be $\Omega=\{0,1\}$ and the CSP $\Psi$ would be composed of the single predicate $\vee_{3}:\{0,1\}^3 \rightarrow \{0,1\}$. This predicate is just the logical OR of the three input variables. Any 3CNF $\phi$ can have its clauses be expressed as constraints $(S,\vee_{3})$ where $S$ would be the variables inputted into $\vee_{3}$. Hence, an assignment $F$ would be an assignment into the variables found in 3CNF $\phi$, showing that $\mathsf{Opt}(\phi)$ aligns with the definition given in the last section. \newline

      \item For $\ELin$, the domain would be $\Omega = \mathbb{F}_2$ and  $\Psi$ consist of predicates of the form $(x_1,\cdots,x_3) = x_1 + x_2 + x_3$ and $(x_1,\cdots,x_3) = x_1 + x_2 + x_3 + 1$ representing both types of linear constraints found in a system of three-variable equations over $\mathbb{F}_2$. An instance $\xi$ would consist of constraints $(S,\psi)$ where $S$ would be a three variable tuple containing the variables appearing in the linear constraint $\psi$. \newline

      \item For $\mathsf{MaxCut}$, the domain can be defined as $\Omega = \{-1,1\}$ with $\Psi$ set to the simple predicate $\neq:\{-1,1\}^2 \rightarrow \{0,1\}$. This simply tests if the inputted values are not equal. The variable set $V$ of an instance $\I$ would be indexed by the vertices contained in a graph $G=(V,E)$. There is one constraint tuple, $((v_i,v_j), \neq)$ for every edge $(v_i,v_j) \in E$. Thus, an assignment $F:V \rightarrow \{-1,1\}$ would assign the vertices into two partitions such that a constraint is satisifed iff the corresponding edge is contained in $\mathcal{C}$. \newline

      \item The CSP for $\mathsf{Max3Color}$, the maximization counterpart for the $\NP$-complete decision problem, $\mathsf{3Color}$, is similarly defined to that of $\mathsf{MaxCut}$ except our domain would be $\Omega = \{0,1,2\}$. This signifies the three possible colors to color any vertex $v \in V$ in an input graph $G = (V,E)$.
    \end{itemize}
  \end{example}

  As implied in the examples above, there is a generic method to formulate a maximization problem in respect to a given CSP $\Psi$:
  %
  \begin{definition}
    For a given CSP $\Psi$, formulate $\mathsf{MaxCSP}(\Psi)$ as the problem: given an instance $\I$, output an assignment $F$ which satisfies the largest number of constraints in $\I$.
  \end{definition}

Note that Definition \ref{Opt3SATDef} for a $\beta$-approximation algorithm extends to this maximization problem in the natural way.

\subsection{Gap Problems}

The $\NP$-hardness theory frequently relies on Karp reductions from decision problems to decision problems. In light of this, we can tailor this paradigm to optimization problems in the form of so-called \emph{gap problems}.

\begin{definition} \label{promiseDef}
  A \emph{promise problem} is defined as a tuple $(\mathsf{YES}, \mathsf{NO})$ where $\mathsf{YES},\mathsf{NO} \subseteq \Sigma^*$ in respect to some alphabet $\Sigma$. Furthermore, we require that $\mathsf{YES} \cap \mathsf{NO} = \emptyset$ but not necessarily that $\mathsf{YES} \cup \mathsf{NO} = \Sigma^*$.
\end{definition}

\begin{definition} \label{gapDef}
  Given a $\MaxCSP$ problem, we define $\GapMaxCSP{\alpha}{\beta}$ for $\alpha < \beta$ as the promise problem: given an instance $\I$:
  \begin{align}
      \I \in \mathsf{YES} & \iff \mathsf{Opt}(\I) \geq \beta \\
        \I \in \mathsf{NO} & \iff \mathsf{Opt}(\I) < \alpha
  \end{align}
  Furthermore, an algorithm $A$ decides $\MaxCSP$ if for input instance $\I$ it accepts iff $\I \in \mathsf{Yes}$ and rejects iff $\I \in \mathsf{No}$. If $\I \not\in \mathsf{Yes} \cup \mathsf{No}$, we do not care what the algorithm outputs.
\end{definition}

In particular, we deem a $\GapMaxCSP{\alpha}{\beta}$ problem as $\NP$-hard if for every language $L \in \NP$, there exists a polynomial-time reduction $f$ taking input strings $x \in \{0,1\}^*$ to CSP $\Psi$ instances such that:
\begin{align*}
  x \in L \implies \mathsf{Opt}(f(x)) & \geq \beta \\
  x \not\in L \implies \mathsf{Opt}(f(x))&  < \alpha
\end{align*}

The $\NP$-hardness of approximation algorithms reduces to that of gap problems as shown in the below observation:

\begin{theorem} \label{GapCSPtoAlgHard}
Suppose $\GapMaxCSP{\alpha}{\beta}$ is $\NP$-hard for CSP $\Psi$, then approximating $\MaxCSP$ to at least an $\frac{\alpha}{\beta}$ factor is $\NP$-hard.
\end{theorem}
%
\begin{proof}
Suppose there exists an algorithm $A$ which can $\frac{\alpha}{\beta}$-approximate $\MaxCSP$. For an instance $\I$ such that $\mathsf{Opt}(\I) \geq \beta$:

\begin{gather*}
A(\I) \geq  \frac{\alpha}{\beta} \cdot \mathsf{Opt}(\I) = \frac{\alpha}{\beta} \cdot \beta =  \alpha \\
\end{gather*}

Else if $\mathsf{Opt}(\I) < \alpha$
\begin{gather*}
A(\I) \leq \mathsf{Opt}(\I) < \alpha
\end{gather*}
by Definition \ref{Opt3SATDef} adapted to $\MaxCSP$ instances. Hence, the algorithm can decide  $\GapMaxCSP{\alpha}{\beta}$ by checking its outputted value in respect to $\alpha$.
\end{proof}
%
This in particular demonstrates that showing the hardness of approximating a particular problem is equivalent to showing the hardness of its corresponding gap problem.

%It was shown early on that traditional Karp reductions were unlikely to produce the gaps required in Definition \ref{gapDef} in reducing $\SAT$ to such a promise problem.


%Traditional Karp reductions don't work

\section{The PCP Theorem}
\subsection{Intuitions}
This section will be centered around the seminal PCP Theorem \cite{arora1998proof}, \cite{arora1998probabilistic}, which characterized $\NP$ in a framework considered unconventional at the time. PCPs, or Probabilistically Checkable Proofs, represent a twist on the idea of $\NP$. Recall that $\NP$ roughly represents the languages which have verifiers which can check proofs of membership in polynomial time. PCPs represent an extension of this definition where the verifier can be \emph{probabilistic} and is granted \emph{random access} to the proof string $\pi$. If we allow the verifier to simply query $\pi$ by outputting a index $i$, it has access to $\pi[i]$. Since we can express an index in $\log{n}$ bits, this in theory gives the verifier access to proof strings of exponential length. To formalize these notions, we begin with definitions:

\begin{definition} \label{pcpdef}
  Given a language $L$ and $r,q: \mathbb{N} \rightarrow \mathbb{N}$, a \emph{(r(n),q(n))-$\PCP$-verifier} for $L$ consists of a polynomial-time algorithm $V$ with the following properties: \newline

  \begin{itemize}
    \item For input strings $x \in \{0,1\}^n, \; \pi \in \{0,1\}^{\leq N}$ for $N = q(n)2^{r(n)}$, $V$ makes $r(n)$ coin flips and decides $q(n)$ queries addresses $i_1, \cdots, i_{q(n)}$ of the proof $\pi$. Based on these queries, it outputs $1$ for ``accept" or $0$ for ``reject". \newline

    \item (Completeness) For $x \in L$, there exists some proof $\pi$ such that $V(x,\pi,r) = 1$ for all random coin tosses $r$. In other words:
    %
    \begin{equation}
      \mathbb{P}_{r}[V(x,\pi,r) = 1] = 1
    \end{equation}

    \item (Soundness) For $x \not\in L$, for all proofs $\pi$:
    %
    \begin{equation}
      \mathbb{P}_{r}[V(x,\pi,r) = 1] \leq \frac{1}{2}
    \end{equation}
  \end{itemize}
  Define the class $\PCP(r(n),q(n))$ as the set of languages $L$ which has a $(c\cdot r(n),d\cdot q(n))$-$\PCP$-verifier for some constants $c,d > 0$.
\end{definition}

\begin{remark}
  Sometimes the completeness criterion is too strong for our purposes (see the comments on $\ELin$ problem). In these cases, we like to denote the class $\PCP_{\alpha, \beta }(r(n),q(n))$ as the languages $L$ which have a $(r(n),q(n))$-$\NP$-verifier such that the completeness and soundness criteria are amended as below: \newline

  \begin{itemize}
    \item (Completeness) For $x \in L$, there exists some proof $\pi$ such that $V(x,\pi,r) = 1$ for all random coin tosses $r$. In other words:
    %
    \begin{equation}
      \mathbb{P}_{r}[V(x,\pi,r) = 1] \geq \beta
    \end{equation}

    \item (Soundness) For $x \not\in L$, for all proofs $\pi$:
    %
    \begin{equation}
      \mathbb{P}_{r}[V(x,\pi,r) = 1] < \alpha
    \end{equation}
  \end{itemize}
Here, $\beta$ is the \emph{completeness parameter} while $\alpha$ is the \emph{soundness parameter}. The class introduced in the original definition would thus be denoted as $\PCP_{\frac{1}{2},1}(r(n),q(n))$. PCP verifiers whose completeness parameter is one ($\beta = 1$) is deemed as \emph{perfectly complete}.
\end{remark}

The PCP Theorem says that $\NP$ is \emph{exactly} the class of PCPs which uses a \emph{logarithmic} number of random bits and a \emph{constant} number of queries.
%
\begin{theorem} \label{pcptheorem} (The $\PCP$ Theorem \cite{arora1998proof}, \cite{arora1998probabilistic})
%
\begin{equation}
\NP = \PCP_{\frac{1}{2},1}(O(\log{n}), O(1))
\end{equation}
\end{theorem}

Actually, one direction of this theorem is not too difficult to see:

\begin{proposition}
For every constants $Q \in \mathbb{N}, c > 0$, $\PCP_{\frac{1}{2}, 1}(c\cdot\log{n}, Q) \subseteq \NP$
\end{proposition}

\begin{proof}
Begin with the observation that $\PCP_{\frac{1}{2},1}(r(n), q(n)) \subseteq \NTIME(q(n)2^{r(n)})$. This is justified by the view of an $\NTIME$ machine simulating the verifier by trying all possible coin tosses and queries to the input string $x$ and proof string $\pi$. It can then count all of the accepting paths to determine the probability of acceptance. If $q = O(1)$ and $r = O(\log{n})$, then the right side of the inclusion will be $\NTIME(2^{O(\log{n})}) = \NP$. Here we use the definition:
%
$ \NP = \bigcup_{c \in \mathbb{N}} \NTIME(n^c) $
\end{proof}

\begin{remark}
  The set of queries a PCP verifier makes could be either \emph{adaptive} or \emph{non-adaptive}. Adaptive queries can be dependent on the outcome of previous queries while non-adaptive queries are independent of one another. The verifiers in these notes will all be non-adaptive for the sake of presentation. The $\PCP$ Theorem still holds when the verifier makes adaptive queries. The only change would be that the proof length would be at most $2^{r(n) + q(n)}$ rather than at most $q(n)2^{r(n)}$.
\end{remark}


%
\subsection{Equivalence of PCP Theorems}
It may be difficult to understand the importance of the PCP Theorem in its form presented in Theorem \ref{pcptheorem}. It turns out there are other equivalent forms of the PCP Theorem more palatable in the context of our goal to prove hardness-of-approximation results.

\begin{theorem} \label{pcpgapsat}($\PCP$ Theorem: $\mathsf{Gap3SAT}$-hardness)
The problem $\mathsf{Gap}_{\alpha,1}\mathsf{Max3SAT}$ is $\NP$-hard. In other words there exists a constant $\alpha  < 1$ such that, for every $\NP$ language $L$, there exists a polynomial-time reduction $f$ mapping $L$ to 3SAT formulas such that:

\begin{align*}
  x \in L & \implies \mathsf{Opt}(f(x)) = 1 \\
  x \not\in L & \implies \mathsf{Opt}(f(x)) \leq \alpha
\end{align*}
\end{theorem}

An immediate consequence of Theorem \ref{pcpgapsat} and Theorem \ref{GapCSPtoAlgHard} is that if there exists an $\alpha$-approximation algorithm for $\mathsf{Max3SAT}$, then $\P = \NP$. With this, we have the first steps towards an inapproximability result: if $\P \neq \NP$, there exists no efficient $\mathsf{Max3SAT}$ algorithm which can approximate better than an $\alpha$ factor. Note that we haven't actually found a concrete value for $\alpha$ yet. This will be addressed once we prove H\aa stad's 3-bit PCP for $\NP$ in a future section.


\begin{theorem} \label{pcptheoremgapcsp}  ($\PCP$-Theorem: $\mathsf{GapMaxqCSP}$- hardness) There exists constants $q \in \mathbb{N}$ and $\alpha < 1$ where the problem $\mathsf{Gap}_{\alpha,1}\mathsf{MaxqCSP}$ is $\NP$-hard. To elaborate, for every $\NP$ language $L$, there exists a polynomial time reduction mapping an $L$ to a instance $f(x)$ of some CSP $\Psi$ over domain $\Omega=\{0,1\}$ where $\Psi$ consists of $q$-ary predicates, such that

\begin{align*}
  x \in L & \implies \mathsf{Opt}(f(x)) = 1 \\
  x \not\in L & \implies \mathsf{Opt}(f(x)) \leq \alpha
\end{align*}
\end{theorem}

\begin{theorem}
  All the PCP Theorems above are equivalent to one another.
\end{theorem}

Before we embark on the proof, let us establish an equivalence between PCPs and CSPs:

\begin{lemma} \label{equipcpcsp} (Equivalence between PCPs and CSPs)
  Theorem \ref{pcptheorem} and  Theorem \ref{pcptheoremgapcsp} are equivalent.
\end{lemma}

\begin{proof}
First, assume $\NP = \PCP_{\frac{1}{2},1}(O(\log{n}), O(1))$. We will outline a procedure to convert the verifier $V$ into an instance $\I$ for a $q$-ary CSP $\Psi$ for some constant $q$. For some input string $x \in \{0,1\}^n$ and proof string $\pi$, let $r \in \{0,1\}^{c \cdot \log{n}}$ be the coin flips made by $V$ and $V_{x,r}$ be the deterministic procedure which is executed on input $x$ and coin flip $r$ such that $V_{x,r} = 1$ iff $V$ accepts proof $\pi$ on input $x$ and coin flip $r$. We can define the domain of our constructed CSP $\Psi$ to be $\Omega=\{0,1\}$ and the predicates to be $\{V_{x,r}\}_{r}$. Now our instance $\I$ of $\Psi$ is casted as the tuples $(S, V_{x,r})$ where $S$ will be at most a $q$-sized tuple indicating which indices of the proof $\pi$ are queried when conditioned on $r$. This yields a polynomially-sized $\mathsf{qCSP}$ instance $\I$. Furthermore, since the verifier $V$ runs in polynomial time, it's execution can be simulated on all $r$ to output the instance $\I$ in polynomial-time. Thus, we have given a polynomial-time reduction from an input $x$ to its corresponding CSP instance $\I$, so Theorem \ref{pcptheoremgapcsp}. \newline

Conversely, suppose we had a reduction from $\NP$ to $\mathsf{Gap}_{\alpha,1}\mathsf{MaxqCSP}$ as stated in Theorem \ref{pcptheoremgapcsp}. We devise polynomial-time reduction taking an instance $\mathsf{MaxqCSP}$ to a polynomial-time PCP verifier $V$ using logarithmic number of random bits and a constant number of queries to the supplied proof $\pi$. For an input $x \in \{0,1\}^n$, the proof will be expected to be an assignment to its respective instance $f(x)$ in the notation utilized in Theorem \ref{pcptheoremgapcsp}. Verifier $V$ makes coin flips $r \in \{0,1\}^{c\log{n}}$ to choose one constraint tuple $(S,\psi)$ where $\psi$ is some $q$-ary predicate. Only a logarithmic number of random bits are required to query any constraint in instance $f(x)$ as the polynomial-time $\NP$ reduction can only generate a polynomial number of such constraints.  The PCP only has to make $q$-queries to the proof $\pi$ to find the assignments to the variables listed in $S = (v_1,\cdots,v_q)$. By the properties listed in Theorem \ref{pcptheoremgapcsp}, the PCP verifier $V$ must have completeness $1$ and soundness $\leq \frac{1}{2}$ as claimed.
\end{proof}

\begin{lemma} (Equivalnce between $\mathsf{GapMaxqCSP}$and $\mathsf{GapMax3SAT}$ Theorem \ref{pcpgapsat} and Theorem \ref{pcptheoremgapcsp} are equivalent.
\end{lemma}
%
\begin{proof}
Since any $\mathsf{3Sat}$ instance can be seen as a particular type of $\mathsf{3CSP}$ instance, one direction is immediate. Conversely, if we assume Theorem \ref{pcptheoremgapcsp}, we aim to find some constant $\alpha'$ such that there exist a reduction from an instance of $\mathsf{Gap}_{1-\alpha,1}\mathsf{MaxqCSP}$ for $q \in \mathbb{N}$ claimed in Theorem \ref{pcptheoremgapcsp} to an instance of $\mathsf{Gap}_{1-\alpha',1}\mathsf{Max3Sat}$.
%
Let the CSP $\Psi$ be comprised $q$-ary predicates $\psi$. For an instance $\I$ of $\Psi$, an constraint tuple $(S,\psi)$ can be expressed as a logical AND of $2^q$ clauses where each clause is a logical OR of $q$ variables or their negations. In other words, $(S,\psi)$ is essentially a CNF of width $q$ and of size at most $2^q$. We can then construct an ``equivalent" 3CNF $\psi'_S$ as follows: add extra symbols $\Pi_{1}, \cdots, \Pi_{(q-3)2^{q}}$. It can be shown that there exists a 3CNF $\psi'_S$ of size at most $(q-2)2^{q}$ such that:

\begin{enumerate}
  \item For every $x \in \{0,1\}^q$ which causes $\psi(x) = 1$, there exists an assignment $\Pi$ such that $\psi'(x,\Pi) = 1$.
  \item Else if $\psi(x) = 0$, then for all assignments $\Pi$, $\psi'(x,\Pi) = 0$
\end{enumerate}

Now take the total 3CNF defined by a conjection of all such $\psi'_S$:
\[ \psi_{\I} = \bigwedge_{(S,\psi) \in \I}  \psi_S' \]

The total formula $\psi_\I$ is determined by at most $mq2^q$ number of clauses and at most $n + m(q-3)2^q$ number of variables.
%
If an instance $\I$ has all of its constraints satisfiable by some assignment, then by proprety (1) listed above, there must exist a assignment to the variables of the $\psi_{\I}$ such that it is satisifed.
%
On the other hand, if for all assignments to instance $\I$ only satisfy at most an $1 - \alpha$ fraction of constraints, then fraction of clauses satisfied can be at most $1 - \alpha + \alpha(1 - \frac{1}{(q-2)2^qa}) = 1 - \frac{\alpha}{(q-2)2^q}$. This is due to the fact that for each unsatisfied $\psi_S'$, the most number of its clauses which can be satisfied by any assignment will be $(q-2)2^q - 1$. Taking $\alpha' = \frac{\alpha}{(q-2)2^q}$
yields the required constant.
\end{proof}

\begin{remark}
  The existence of the ``equivalent" 3CNF is outlined in Problem 7.11 of the O'Donnell textbook \cite{o2014analysis}.
 \end{remark}

\begin{remark}
The proof of the PCP Theorem (Theorem \ref{pcptheorem}) is quite involved and out of the scope of these notes. The original proof hinged on clever algebraic techniques such as low-degree testing \cite{arora1998probabilistic}. A simpler combinatorial proof by Dinur arrived years later  \cite{dinur2007pcp}. For more information, refer to the cited publications or Chapter 22 of the Arora, Barak textbook \cite{arora2009computational}.
\end{remark}


%2 The PCP Theorem: An Introduction (DIMACS)

%\subsection{Examples of PCPs}


%PCPs and CSPs.
%\subsection{``Naive" PCP for NP}

\section{Label-Cover and Projection Games}
%Definition of Label-Cover
We now introduce a problem which manages to provide a natural paradigm for capturing the essence of CSPs and proving inapproximability results. These ``projection games" were introduced by Bellare, Goldreich, and Sudan \cite{bellare1998free}. The $\NP$-hardness of the gap problem version of Label Cover was used by H\aa stad to show tight inapproximability results for $\mathsf{Max3SAT}$ and $\ELin$ \cite{haastad2001some}.

\begin{definition}
A \emph{Label Cover (LC) Problem} instance $\mathcal{G}$ is defined by a bipartite graph $(A \sqcup B,E)$, finite alphabets $\Sigma_A, \Sigma_B$, and a set of projections $\pi_e:\Sigma_A \rightarrow \Sigma_B$ for every edge $e \in E$. Define an \emph{assignment} as consisting of two maps $\mathfrak{A}: A \rightarrow \Sigma_A$, $\mathfrak{B}: B \rightarrow \Sigma_B$. An edge $e = (a,b) \in E$ is said to be satisfied by this assignment if the assignment is compatible with projection $\pi_e$:

\begin{equation}
  \pi_e(\mathfrak{A}(a)) = \mathfrak{B}(b)
\end{equation}

The value of this game will be

\begin{equation} \label{optvalLC}
  \mathsf{Opt}(\mathcal{G}) = \max_{(\mathfrak{A},\mathfrak{B})} \mathbb{E}_{e \sim E}[e \text{ satisfied}]
\end{equation}
In other words, the value will be the largest fraction of edges satisfied by any assignment to the vertices. The corresponding gap problem for Label Cover, $\mathsf{Gap}_{\alpha,\beta}\mathsf{LC}$, is defined as the promise problem:
%
\begin{align*}
    \mathsf{YES} & = \{\mathcal{G} \mid \mathsf{Opt}(\mathcal{G}) \geq \beta\} \\
    \mathsf{NO} & = \{\mathcal{G} \mid \mathsf{Opt}(\mathcal{G}) < \alpha \}
\end{align*}

In the case of perfect completeness, we abbreviate $\mathsf{Gap}_{\alpha,1}\mathsf{LC}$ as simply $\mathsf{Gap}_{\alpha}\mathsf{LC}$.
\end{definition}

There are a few observations worthy of mentioning here. The first regards a type of equivalence between CSP instances and Label Cover instances.
%
Specifically, let $\I$ be an instance of a given $r$-ary CSP $\Psi$ over domain $\Omega$. We can translate this CSP instance into a Label Cover instance as follows:
%
Let the left-hand partition $A$ of our bipartite graph be indexed by the set of constraints $(S,\psi)$ and the right-hand partition $B$ be indexed by the variables of the CSP $V$.
%
Draw an edge from a constraint tuple $(S,\psi)$ to a variable $v$ if that variable appears in $S$.
%
Set $\Sigma_A = \{(q_1,\cdots,q_r) \in \Omega^r \mid \exists \psi \in \Psi, \; \psi(q_1,\cdots,q_r) = 1 \}$
%
and $\Sigma_B = \Omega$, and for every edge $e = ((v_1,\cdots,v_r), \psi), v)$ define the projection $\pi_e:\Sigma_A \rightarrow \Sigma_B$ to be

\[ \pi_e(\omega_1, \cdots, \omega_r) =  \omega_i \text{ if } v_i = v\] \newline

%Finish this.

On the other hand, every Label Cover instance can be seen as a $2$CSP over a sufficiently large domain: the predicates of the CSP would be all $2$-ary predicates $\pi:\Sigma_A \times \Sigma_B \rightarrow \{0,1\}$ representing every possible map from $\Sigma_A \rightarrow \Sigma_B$. Thus, the domain of our CSP can be defined as $\Omega = \Sigma_A \cup \Sigma_B$. The corresponding instance of this CSP would be $(S,\pi_e)$ where $\pi_e$ represents the predicate corresponding to the edge $e$'s projection map $\pi_e$ and $S = (a,b)$ would be the vertices of $e$ between $A$ and $B$ respectively. \newline

By Theorem \ref{equipcpcsp} and the observations made above, there is a method of converting the PCP for $\NP$ given by the PCP Theorem to a CSP, then converting that CSP into a Label Cover instance. This shows that $\mathsf{Gap}_\alpha\mathsf{LC}$ for some $\alpha > 0$ must be $\NP$-hard:

\begin{theorem} (Weak Projection Games Theorem)
 $\mathsf{Gap}_\alpha\mathsf{LC}$ for some $\alpha > 0$ is $\NP$-hard.
\end{theorem}

The primary drawback of this theorem is that it does give us an $\NP$-hardness result for \emph{every} constant $\alpha > 0$. This will be addressed by Raz's Parallel Repetition Theorem.
%examples.
%trivial NP hardness proof.

%\subsection{}
%Arora, Barak

\subsection{Raz's Parallel Repetition Theorem}
In this section, we give a cursory outline of Raz's Parallel Repetition Theorem and its consequences. First, we can extend Definition \ref{pcpdef} to include proof strings over non-boolean alphabets.
%
To accomodate such verifiers, we can extend the definition to $\PCP^\Sigma_{\beta, \alpha}(r(n),q(n))$, the class of languages with polynomial-time verifiers taking proof strings $\pi$ over non-boolean alphabet $\Sigma$ and uses $r(n)$ random bits and at most $q(n)$ queries to $\pi$.
%
The following theorem shows that one can reduce the number of queries to two at the expense of a weaker soundness constant and a larger alphabet:
%
\begin{theorem} \label{2queryinclusion}
  $\PCP^\Sigma_{1 - \alpha, 1}(r(n),q(n)) \subseteq \PCP^{\Sigma^q}_{1 - \frac{\alpha}{q},1}(r(n) + \log{q(n)}, 2)$
\end{theorem}
%
\begin{proof}
Given a verifier for a language $L \in \PCP^\Sigma_{\beta, 1 - \alpha}(r(n),q(n))$, it behaves as such on input $x \in \{0,1\}^n$:
%
\begin{enumerate}
  \item Flips a $r(n)$ coins which we denote as $R$.
  \item Using $R$, it computes a series of indices $i_1, \cdots i_q$ where $q = q(n)$.
  \item It queries $\pi_{i_1}, \cdots \pi_{i_q}\in \Sigma$ from the proof $\pi$.
  \item Finally, it feeds the symbols in a predicate $V_{x,R}(\pi_{i_1}, \cdots \pi_{i_q})$ which outputs ``accept" or ``reject".
\end{enumerate}
We extend this to a modified verfier $V'$ to accomodate the concatenation of two proof strings: if $m = |\Sigma|$ $\pi_1:[m]^q$ then $\rightarrow \Sigma^{q}$ and $\pi_2:[m]\rightarrow \Sigma$:
%
\begin{enumerate}
  \item Flips a $r(n)$ coins which we denote as $R$.
  \item Using $R$, it computes a series of indices $i_1, \cdots i_q$ where $q = q(n)$
  \item Queries $\pi' = \pi_1(i_1, \cdots, i_q)$
  \item Computes the predicate:
  \[ V_{x,R}'(\pi') = V_{x,R}(\pi'_1,\cdots,\pi'_q) \]
  \item Finally, chooses random $\ell \in [q]$ and checks if $\pi_2(i_\ell) = \pi'_\ell$.
  \item The verifier accepts iff both checks pass.
\end{enumerate}

For completeness, if there exists a proof $\pi$ which the original verifier $V$ accepts with probability one, then setting $\pi_1(i_1, \cdots, i_q) = (\pi_{i_1},\cdots,\pi_{i_q})$ and $\pi_2 = \pi$ will induce $V'$ to accept with probability one by construction.
%
As for soundness, suppose  at least $\alpha$ fraction of the predicates $V_{x,R}$ reject in respect to some proof string $\pi$. Each one of the predicates contained in this fraction must fail the check at Step $4$ if $\pi' = \pi_1(i_1, \cdots, i_q) = (\pi_{i_1},\cdots,\pi_{i_q})$. However, $\pi_1$ does not have to respect this rule for any $q$-tuple of queries $(i_1, \cdots, i_q)$ given as input. If so, there must exist at least one index $\ell$ such that $\pi'_\ell \neq \pi_2(i_\ell) = \pi_\ell$. Since such an index is chosen with probability $\frac{1}{q}$, we have that $V'$ must reject with probability at least $\frac{\alpha}{q}$. This completes the proof.
\end{proof}


%
Raz's Parallel Repetition Theorem constructed a verifier which allowed the soundness constant to drop exponentially with the cost of bloating the alphabet size:
\begin{theorem} (Raz's Parallel Repetition Theorem \cite{raz1998parallel}) \label{razparallelrep}
  For all $s \in (0,1)$, there exists $c_s \in (0,s)$ such that:
    \[ \PCP^{\Sigma}_{s,1}(r,2) \subseteq \PCP^{\Sigma^t}_{c_s^t,1}(rt,2)  \]
\end{theorem}

\begin{corollary} \label{razcorollary} For all $\epsilon > 0$, there exists an alphabet $\Sigma$ such that $|\Sigma| \leq \mathsf{poly}(\frac{1}{\epsilon})$:
  $$ \NP \subseteq \PCP^{\Sigma}_{\epsilon,1}(O(\log{n}\cdot\log{1/\epsilon}),2) $$
\end{corollary}
%
\begin{proof}
  For some constant $Q > 0$,
  \begin{align*}
    \NP & \subseteq \PCP_{\frac{1}{2},1}^{\{0,1\}}(O(\log{n}), Q) \\
    & \subseteq \PCP_{1- \frac{1}{2Q},1}^{\{0,1\}^Q}(O(\log{n}), 2) \\
    & \subseteq \PCP_{c_Q^t,1}^{\{0,1\}^{Qt}}(O(t\log{n}), 2) \\
    & \subseteq   \PCP_{\epsilon,1}^{\Sigma}(O(\log{n}\cdot\log{1/\epsilon}), 2)
  \end{align*}
  where $|\Sigma| \leq \mathsf{poly}(\frac{1}{\epsilon})$. The second inclusion follows from Theorem \ref{2queryinclusion} and the third inclusion follows from Theorem \ref{razparallelrep}.
\end{proof}


By combining Corollary \ref{razcorollary}, the $\NP$-hardness of Label Cover comes to fruition:

\begin{theorem} (Projection Games Theorem) \label{labelcoverhard}
  For every $\epsilon > 0$, there exist alphabets $\Sigma_A, \Sigma_B$ where $|\Sigma_A|,|\Sigma_B| \leq \mathsf{poly}(\frac{1}{\epsilon})$ such that $\mathsf{Gap}_\epsilon\mathsf{LC}$ is $\NP$-hard.
\end{theorem}

This is shown by embedding the query pairs of the verifier into the projection constraints of a Label Cover instance graph.

\section{H\aa stad's 3-bit PCP}
%Arora, Barak
\subsection{Inapproximability Results for $\ELin$ and $\mathsf{Max3Sat}$}
The significance of H\aa stad's PCP for $\NP$ is its use in showing tight inapproximability results for $\ELin$ and hence $\mathsf{Max3Sat}$. The theorem is first stated below:

\begin{theorem}(H\aa stad's 3-bit PCP, \cite{haastad2001some})
For every $\delta > 0$ and $L \in \NP$, there exists a PCP verifier for $L$ over the boolean alphabet such that for every
\begin{itemize}
  \item The verifier $V$ queries 3 bits of the proof $x_{q_1},x_{q_2}, x_{q_3} \in \pi$ such that verification predicate is a three variable linear equation over $\mathbb{F}_2$ depending on the queried bits $x_{q_1},x_{q_2}, x_{q_3}$.
  \item If $x \in L$, then there exists a proof $\pi$:
        \begin{equation}
          \mathbb{P}[V(x,\pi) = 1] \geq 1 - \delta
        \end{equation}
  \item If $x \not\in L$, then for all proofs $\pi$:
        \begin{equation}
          \mathbb{P}[V(x,\pi) = 1] \geq \frac{1}{2} + \delta
        \end{equation}
\end{itemize}

\end{theorem}

\subsection{The Long Code}
To be written...

\subsection{Aside on Dictatorship Testing}
The first step towards testing if an input boolean function $f:\{-1,1\}^n \rightarrow \{-1.1\}$ is a dictator arises from the Blum-Luby-Rubinfeld Test (BLR) which we restate below: \newline

\begin{enumerate}
  \item Sample $x,y \sim_R \{-1,1\}^n$
  \item Accept iff $f(x)f(y) = f(xy)$
\end{enumerate}

\begin{theorem}
  Suppose the BLR test accepts $f:\{-1,1\}^n \rightarrow \{-1,1\}$ with probability $1 - \epsilon$, then $f$ is $\epsilon$-close to a linear function $\chi_{S*}$ for some $S^* \subseteq [n]$.
\end{theorem}
%
Certainly the dictators $\chi_i, \; i \in [n]$ are linear functions. Hence, if $f = \chi_i$ for some $i$, then the BLR test accepts with probability one. However, we have the rest of the parity functions $\chi_{S}, |S| \geq 2$ which the BLR Test cannot distinguish. We need to amend the test to ensure that parity functions of higher weight are rejected with high probability. H\aa stad proposed modifiying the vanilla BLR test to add noise to the sampled product $xy$. Although this sacrifices perfect completeness, it penalizes large parity functions: \newline
%
\begin{enumerate}
  \item Sample $x,y \sim \{-1,1\}^n$
  \item Sample $z \sim N_{1-2\epsilon}(xy)$
  \item Accept iff $f(x)f(y)=f(z)$
\end{enumerate}
%

\begin{lemma} (Completeness of the Noisy BLR Test)
 If $f = \chi_{i}$ is a dictator for some $i$, then
  \[ \mathbb{P}[\text{ Noisy BLR test accepts }] \geq 1 - \epsilon \]
\end{lemma}
\begin{proof}
  Note that if $z \sim N_{1-2\epsilon}(xy)$ for some $x \in \{-1,1\}^n$, then $y$ can be expressed as below:
  \begin{equation}
    z_i = \begin{cases}
             \phantom{-} x_iy_i \text{ with probability } 1- \epsilon \\
             - x_iy_i \text{ with probability } \epsilon
          \end{cases}
  \end{equation}
  By the acceptance criterion,
  \[ f(z) = f(x)f(y) \implies z_i = x_iy_i \]
  This occurs with probability $1- \epsilon$ by the observation above, yielding completeness as claimed.
\end{proof}
%
%
\begin{lemma} Suppose that for some constant $\nu > 0$:
\begin{equation*}
  \mathbb{P}[\text{ Noisy BLR test accepts }] \geq \frac{1}{2} + \nu
\end{equation*} then
\begin{equation}
  2\nu \leq \sum_{S \subseteq [n]} \hat{f}(S)^3(1-2\epsilon)^{|S|}
\end{equation}
\end{lemma}
%
\begin{proof}
  The proof begins by noticing the accepting probability can be expressed as:
  \[ \mathbb{P}[\text{ Noisy BLR test accepts }] = \frac{1}{2} + \frac{1}{2}\mathbb{E}_{x,y,z} \left[f(x)f(y)f(z) \right] \]
  By our assumption, we prove that:
  \begin{align*}
    \frac{1}{2} + \nu & \leq \frac{1}{2} + \frac{1}{2}\mathbb{E}_{x,y,z} \left[f(x)f(y)f(z) \right] \implies \\[0.7ex]
    2\nu & \leq \mathbb{E}_{x,y,z} \left[f(x)f(y)f(z) \right] \implies \\[0.7ex]
    2\nu & \leq\mathbb{E}_{x,y,z} \left[f(x)f(y) \mathbb{E}_{z \sim N_{1 - 2\epsilon}(xy)} \left[f(z)\right]\right] \implies \\[0.7ex]
    2\nu & \leq \mathbb{E}_{x}\left[ f(x)\mathbb{E}_{y}\left[ f(y)\mathcal{T}_{1-2\epsilon}f(xy)\right]\right] \\[0.7ex]
    2\nu & \leq \mathbb{E}_{x}\left[ f(x) (f * \mathcal{T}_{1-2\epsilon}f)(x)\right] \implies\\[0.7ex]
    2\nu & \leq \sum_{S \subseteq} \widehat{f}(S)\widehat{f}(S)\widehat{\mathcal{T}_{1-2\epsilon}f}(S) \implies \\[0.7ex]
    2\nu & \leq  \sum_{S \subseteq} \widehat{f}(S)^3 (1-2\epsilon)^{|S|}
  \end{align*}
  as claimed.
\end{proof}
%
\begin{corollary}
  (Soundness of the Noisy BLR Test) There exists some $S^* \subseteq [n]$ such that
  \begin{equation}
    |\widehat{f}(S^*)| \geq 2\nu \quad |S^*| \leq O\left(\frac{1}{\epsilon}\log{\frac{1}{\nu}}\right)
  \end{equation}
\end{corollary}

\section{Unique Games}
\subsection{Definitions}
The PCP Theorem culminated in a proof of the $\NP$-hardness of Label Cover by H\aa stad. Although these results gave proofs of the $\NP$-hardness of $\mathsf{Gap}_{\frac{7}{8} + \epsilon, 1- \epsilon}$-$\mathsf{Max3SAT}$ and $\mathsf{Gap}_{\frac{1}{2} + \epsilon, 1- \epsilon}$-$\mathsf{MaxE3Lin}$, similar hardness proofs for other canonical problems such as $\mathsf{MaxCut}$ didn't seem to follow from these ideas. In his seminal paper, Khot proposed a relaxation of the Label Cover Problem \cite{khot2002power}. The instances of this relaxed version are called \emph{Unique Games}:

\begin{definition}
  A \emph{Unique Label Cover Problem} with $m$ labels (UniqueLC($m$)) instance $\mathcal{U}$ is defined by a bipartite graph $(A \sqcup B,E)$, finite alphabet $\Sigma_A = \Sigma_B = \Sigma$ such that $|\Sigma| = m$, and a set of \emph{permutations} $\pi_e:[m] \rightarrow [m]$ for every edge $e \in E$. Define an \emph{assignment} as consisting of a map $\sigma: A \sqcup B \rightarrow [m]$. An edge $e = (a,b) \in E$ is said to be satisfied by this assignment if the assignment is compatible with projection $\pi_e$:

  \begin{equation}
    \pi_e(\sigma(a)) = \sigma(b)
  \end{equation}

  The value of this game will be

  \begin{equation} \label{optvalLC}
    \mathsf{Opt}(\mathcal{G}) = \max_{\sigma} \mathbb{E}_{e \sim E}[e \text{ satisfied}]
  \end{equation}
  In other words, the value will be the largest fraction of edges satisfied by any assignment to the vertices. The corresponding gap problem for Label Cover, $\mathsf{Gap}_{\alpha,\beta}\mathsf{UniqueLC}(m)$, is defined as the promise problem:
  %
  \begin{align*}
      \mathsf{YES} & = \{\mathcal{G} \mid \mathsf{Opt}(\mathcal{U}) \geq \beta\} \\
      \mathsf{NO} & = \{\mathcal{G} \mid \mathsf{Opt}(\mathcal{U}) < \alpha \}
  \end{align*}

  In the case of perfect completeness, we abbreviate $\mathsf{Gap}_{\alpha,1}\mathsf{UniqueLC}(m)$ as simply $\mathsf{Gap}_{\alpha}\mathsf{UniqueLC}(m)$.
\end{definition}

\begin{theorem} (Unique Games Conjecture \cite{khot2002power})
  For any constant $\delta > 0$, there exists sufficiently large $m \in \mathbb{N}$ such that $\mathsf{Gap}_{\delta,1-\delta}\mathsf{UniqueLC}(m)$ is $\NP$-hard.
\end{theorem}

%Polynomial time algorithm for guaranteed solutions.

\section{UG-hardness of MAXCUT}
\subsection{Intuitions}
\subsection{Goemans-Williams Algorithm for MAXCUT}

%\section{Semi-definite Programming and Integrality Gaps}


%\section{Quantum Information Theory}
%\section{Other Connections}

\bibliographystyle{amsplain}
\bibliography{biblio}

\end{document}
