\documentclass{amsart}
\usepackage{complexity}

\setlength{\textwidth}{\paperwidth}
\addtolength{\textwidth}{-2in}
\calclayout

\newtheorem{theorem}{Theorem}[section]
\newtheorem{corollary}{Corollary}[theorem]
\newtheorem{lemma}[theorem]{Lemma}

\theoremstyle{definition}
\newtheorem{definition}[theorem]{Definition}
\newtheorem{example}[theorem]{Example}
\newtheorem{xca}[theorem]{Exercise}

\theoremstyle{remark}
\newtheorem{remark}[theorem]{Remark}
\numberwithin{equation}{section}

\theoremstyle{remark}
\newtheorem{notation}[theorem]{Notation}
\begin{document}

\title{Notes on PCPs and Unique Games}

\author{Edward Kim}
\email{ehkim@cs.unc.edu}

\begin{abstract}
We expound on the basic theory of approximating $\NP$ problems. Starting with the PCP Theorem, we introduce background for the Unique Games Conjecture (UGC) and the optimality of the Goeman-Williams Algorithm for MAXCUT assuming the conjecture is true. Finally, we make some connections to Quantum Information Theory, especially Quantum PCPs (qPCPs).
\end{abstract}

\maketitle

%\section{Approximation Algorithms}
%3.2 Limits of Approximation – Introduction (TIFR)
%\subsection{Definitions and Examples}
%\subsection{Gap Problems}

%\section{PCP Theorem}
%2 The PCP Theorem: An Introduction (DIMACS)
%\subsection{Examples of PCPs}
%\subsection{Equivalence of PCP Theorems}
%\subsection{First step PCP for NP}

%\section{Label-Cover and CSPs}
%\subsection{Constraint Satisfaction Problems}
%\subsection{Raz's Parallel Repetition Theorem}
%Arora, Barak
%\subsection{NP-hardness of Label-Cover}

%\section{Hastads 3-bit PCP}
%Arora, Barak
%\subsection{Dictatorship Testing}
%\subsection{Long Code}

%\section{Unique Games}
%\subsection{Definitions}

%\section{UG-hardness of MAXCUT}
%\subsection{Intuitions}
%\subsection{Goemans-Williams Algorithm for MAXCUT}

%\section{Semi-definite Programming and Integrality Gaps}



\end{document}
