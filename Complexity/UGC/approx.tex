\section{Approximation Algorithms and CSPs}
%3.2 Limits of Approximation – Introduction (TIFR)
%AB - Chapter 11
%Explain intuition.
\subsection{Definitions and Examples}
The theory of approximating $\NP$-hard problems roots itself in the following question: ``Is it possible to efficiently approximate $\NP$-complete problem to some arbitrary degree of accuracy?" Since Cook-Levin result demonstrated that the $\SAT$ decision problem is $\NP$-complete, the question could be rephrased as ``Can we find an efficient algorithhm for $\SAT$ which outputs an assignment which satisfies a $1-\delta$ fraction of the clauses for any constant $\delta >  0$?" To this end, let us first introduce a few motivating examples which will find utility in our forthcoming analysis. \newline

Consider $\mathsf{Max3CNF}$, the problem of, given a 3CNF formula $\phi$ as input, outputting an assignment which maximizes the number of clauses satisfied in $\phi$. Given any assignment $x$ to the variables, say as an $n$-bit string while viewing $\phi:\{0,1\}^n \rightarrow \{0,1\}$, there is some $0 \leq \rho_{\phi,x} \leq 1$ representing the fraction of clauses satisfied in $\phi$. Take $\mathsf{Opt}(\phi)$ to be the maximum such value over all such assignments:

\[ \mathsf{Opt}(\phi) = \max_{x \in \{0,1\}^n}\rho_{\phi,x} \]

An polynomial-time algorithm which solves $\MaxCNF$ is one which takes a 3CNF $\phi$ as input and outputs $\Opt(\phi)$. Naturally, $\mathsf{Max3CNF}$ is $\NP$-hard since its corresponding decision problem $\mathsf{3CNF}$ is $\NP$-complete i.e $\phi$ is satisfiable iff $\mathsf{Opt}(\phi) = 1$.
%
However, we can ask if there exists an algorithm $A$ which outputs an assignment which satisfies at least some $\beta\cdot\mathsf{Opt}(\phi)$ fraction of the clauses of $\phi$ for some $\beta \leq 1$. To formalize this: \newline
%
\begin{definition} \label{Opt3SATDef}
For $\beta \leq 1$, a polynomial-time algorithm $A$ is deemed as a $\beta$-approximation algorithm for $\mathsf{Max3CNF}$ if $A(\phi)$ outputs an assignment which satisifies at least $\beta\cdot\mathsf{Opt}(\phi)$ fraction of $\phi$'s clauses for every 3CNF instance $\phi$. More specifically, the algorithm $A$ outputs some value such that:
\begin{equation}
  \beta\cdot\mathsf{Opt}(\phi) \leq A(\phi) \leq \mathsf{Opt}(\phi)
\end{equation}
If $\mathsf{Opt}(\phi) = 1$, then $\phi$ is said to be \emph{satisfiable}.
\end{definition}

A canonical example of a simple approximation algorithm for $\mathsf{Max3CNF}$ is the following scheme: for every variable in $\phi$ choose with uniform probability an assignment from $\{0,1\}$. The probability of any given clause being satisfied by such a random assignment is $\frac{7}{8}$. Thus
%
\[ \mathbb{E}_{x \in \{0,1\}^n}[\text{\# satisfied clauses of $\phi(x)$}] = \frac{7m}{8} \]
%
where $m$ is the number of satisfiable clauses of $\phi$. This gives us a $\frac{7}{8}$-approximation algorithm for $\mathsf{Max3CNF}$. Here's another example:
%
\begin{example} \label{e3linexample}
  Let $\mathsf{MaxE3Lin}$ define the problem of the following form: let $\xi$ be defined as a system of linear equations over the field $\mathbb{F}_2$ where every equation contains \emph{exactly} $3$ variables from $n$ variables $x_1,\cdots,x_n$. Find an assignment to the variables which maximizes the number of satisified linear equations in $\xi$. An example is the following system over variables $x_1,x_2,x_3,x_4$:
  %
  \begin{equation*}
    \begin{alignedat}{3}
      x_1 & +{} &  x_2 & +{} & x_3 & = 0 \\
      x_1 & +{} &  x_2 & +{} &  x_4 & = 1 \\
      x_1 & +{} & x_3 & +{} & x_4 & = 1
\end{alignedat}
\end{equation*}
\end{example}
%
The $\mathsf{E3Lin}$ represents a linear system of $\mathsf{E}$xactly $\mathsf{3}$ variables. We can extend the notation treated above for $\mathsf{Max3CNF}$ to this situation. In other words, if $\rho_{\xi,x}$ is the fraction of satisfied linear constraints of $\xi$ under assignment $x \in \mathbb{F}_2^n$.
%
\[\mathsf{Opt}(\xi) =\max_{x \in \{0,1\}^n}\rho_{\xi,x}\]
%
A comment regarding this problem. Firstly, if the given E3Lin instance has a guaranteed solution i.e $\mathsf{Opt}(\phi) = 1$, then Gaussian Elimination will always output an assignment which satisfies all linear constaints in polynomial-time. It's more interesting to consider instances where no such solution satisfing all of the constraints exists over $\mathbb{F}_2^n$. In these cases, $\mathsf{Opt}(\xi) = 1 - \epsilon$ for some $\epsilon \leq 1$. This problem also has a fairly simple $\frac{1}{2}$-approximation algorithm: Set all $x_1 = \cdots = x_n  = 0$ or $x_1 = \cdots = x_n  = 1$ depending on which satisfies the most constraints. This scheme must satisfy at least $\frac{1}{2}$ of the constraints for any instance $\xi$. \newline

\begin{example} \label{maxcutexample}
The problem $\MaxCut$ will be casted as follows: given an undirected graph $G=(V,E)$, find the largest cut of $G$. There is a straightforward LP formulation for this problem:

\begin{align}
  &  \max \sum_{(u,v) \in E} e_{u,v} \\
  & e_{u,v} \leq
  \min \begin{cases}
      x_1 + x_2 \\
      2 - (x_1 + x_2)
  \end{cases} \\
  & e_{u,v} \in [0,1] \\
  & x_v \in [0,1]
\end{align}
%
where the $e_{u,v}$ variables represent the edges $e \in E$ of the input graph and the $x_v$ represent the vertex variables $v \in V$. This will actually be a relaxation from its respective integer program in which the variables can take the integer values:

\begin{align}
  e_{u,v} = \begin{cases}
              1 \text{ if } e_{u,v} \in \mathcal{C} \\
              0 \text{ otherwise }
            \end{cases}
  x_v = \begin{cases}
              0 \text{ if } v \text{ is in partition } S \\
              1 \text{ otherwise }
        \end{cases}
\end{align}

where $\mathcal{C} \subseteq E$ is the edges constituting the cut and $(S,\bar{S})$ is the partition defining the cut. Naturally, a solution to the Integer program will also be a solution to the LP above, which implies that

\[ \mathsf{Opt}_{LP} \geq \MaxCut(G)\]

There actually exists an Semi-definite Programming (SDP) relaxation for $\MaxCut$ called the \emph{Goemans-Williamson} algorithm which will be treated in Section \ref{goemans}.

\end{example}

\subsection{Constraint Satisfaction Problems}

 Generally speaking, it's useful to concretize these problems into a unified framework. The examples presented in the last section have some common interpretation shared between them: they could all be seen as manifestations of Constaint-Satisfaction Problem (CSP):
%
\begin{definition}
  Let $\Omega$ be a finite set deemed as the \emph{domain}. A constraint satisfaction problem (CSP) $\Psi$ over domain $\Omega$ is a finite set of predicates $\psi:\Omega^r \rightarrow \{0,1\}$ where $r$ would be the \emph{arity} of predicate $\psi$. The predicates can be of different arities. The arity of the CSP $\Psi$ would be the maximum of all the arities of the predicates in $\Psi$. \newline

  \noindent An \emph{instance} $\mathcal{I}$ over CSP $\Psi$ is a set of tuples $(S,\psi)$ where if $r$ is the arity of predicate $\psi$, $S = (v_1,\cdots,v_r)$ is some ordered tuple of variables taken from finite set $V$ consisting of variables contained in CSP $\Psi$. These tuples are called the \emph{constaints} of $\mathcal{I}$. In addition, we add the conditon that every variable show up in at least one constraint. Variables which do not appear in any of the constraints can simply be removed. \newline

  \noindent Now given some instance $\mathcal{I}$, an \emph{assignment} $F:V \rightarrow \Omega$ is simply some  map between the variables and the domain. We say that $F$ satisfies a constaint $(S,\psi)$ if $\psi(F(S)) = 1$. For tuple $S = (v_1, \cdots, v_r)$, we define $F(S) = (F(v_1), \cdots, F(v_r))$. Consider the fraction of constraints of satisfied by $F$ in $\mathcal{I}$:
  %
  \begin{equation}
  \mathsf{Val}_{\I}(F) = \mathbb{E}_{(S, \psi) \sim \I}[\psi(F(S))]
  \end{equation}
  By taking the maximum fraction over all such assignments:
  %
  \begin{equation}
  \mathsf{Opt}(\I) = \max_{F: V \rightarrow \Omega} \mathsf{Val}_{\I}(F)
  \end{equation}
\end{definition}

  %
  \begin{example} \hfill{}
    \begin{itemize}
      \item For $\MaxCNF$, the domain would be $\Omega=\{0,1\}$ and the CSP $\Psi$ would be composed of the predicate $\vee_{3}:\{0,1\}^3 \rightarrow \{0,1\}$ just taking the logical ORs of the three input variables. Any instance $\phi$ would be composed of $(S,\vee_{3})$ where $S$ would be the variables inputted into $\vee_{3}$. Hence, an assignment $F$ would be an assignment into the variables found in 3CNF $\phi$, showing that $\mathsf{Opt}(\phi)$ aligns with the definition given in the last section. \newline

      \item For $\ELin$, the domain would be $\Omega = \mathbb{F}_2$ and  $\Psi$ consist of predicates of the form $(x_1,\cdots,x_3) = x_1 + x_2 + x_3$ and $(x_1,\cdots,x_3) = x_1 + x_2 + x_3 + 1$ representing both types of linear constraints found in an system of three-variable equations over $\mathbb{F}_2$. An instance $\xi$ would consist of constraints $(S,\psi)$ where $S$ would be a three variable tuple containing the variables showing up in the linear constraint $\psi$. \newline

      \item For $\mathsf{MaxCut}$, the domain can be defined as $\Omega = \{-1,1\}$ with $\Psi$ set to the simple predicate $\neq:\{-1,1\}^2 \rightarrow \{0,1\}$. This simply tests if the inputted values are not equal. The variable set $V$ of an instance $\I$ would be indexed by the vertices contained in a graph $G=(V,E)$. There is one constraint tuple, $((v_i,v_j), \neq)$ for every edge $(v_i,v_j) \in E$. Thus, an assignment $F:V \rightarrow \{-1,1\}$ would encode a partition of $V$ into a cut $C$ with a constraint becoming satisified if the corresponding edge is contained in $C$. \newline

      \item The CSP for $\mathsf{Max3Color}$, the maximization counterpart for the $\NP$-complete decision problem, $\mathsf{3Color}$, is similarly defined to that of $\mathsf{MaxCut}$ except our domain would be $\Omega = \{0,1,2\}$. This signifies the three possible colors to color any vertex $v \in V$ in an input graph $G = (V,E)$.


    \end{itemize}
  \end{example}

  As implied in the examples above, there is a generic method to formulate a maximization problem in respect to a given CSP $\Psi$:
  %
  \begin{definition}
    For a given CSP $\Psi$, formulate $\mathsf{MaxCSP}(\Psi)$ as the problem: given an instance $\I$, output an assignment $F$ which satisfies the largest number of constraints in $\I$.
  \end{definition}

\subsection{Gap Problems}

The $\NP$-hardness theory frequently relies on Karp reductions from decision problems to decision problems. In light of this, we can tailor optimization problems to related promise problems in the form of so-called \emph{gap problems}.

\begin{definition} \label{promiseDef}
  A \emph{promise problem} is defined as a tuple $(\mathsf{YES}, \mathsf{NO})$ where $\mathsf{YES},\mathsf{NO} \subseteq \Sigma^*$ in respect to some alphabet $\Sigma$. Furthermore, we require that $\mathsf{YES} \cap \mathsf{NO} = \emptyset$ but not necessarily that $\mathsf{YES} \cup \mathsf{NO} = \Sigma^*$.
\end{definition}

\begin{definition} \label{gapDef}
  Given a $\MaxCSP$ problem, we define $\GapMaxCSP{\alpha}{\beta}$ for $\alpha < \beta$ as the promise problem: given an instance $\I$:
  \begin{align}
      \I \in \mathsf{YES} & \iff \mathsf{Opt}(\I) \geq \beta \\
        \I \in \mathsf{NO} & \iff \mathsf{Opt}(\I) < \alpha
  \end{align}
  Furthermore, an algorithm $A$ decides $\MaxCSP$ if for input instance $\I$ it accepts iff $\I \in \mathsf{Yes}$ and rejects iff $\I \in \mathsf{No}$. If $\I$ is in neither $\mathsf{Yes}, \mathsf{No}$, we do not care what the algorithm outputs.
\end{definition}

In particular, we deem a $\GapMaxCSP{\alpha}{\beta}$ problem as $\NP$-hard if for every language $L \in \NP$, there exists a polynomial-time reduction $f$ taking $x$ to CSP $\Psi$ instances such that:
\begin{align*}
  x \in L \implies \mathsf{Opt}(f(x)) & \geq \beta \\
  x \not\in L \implies \mathsf{Opt}(f(x))&  < \alpha
\end{align*}

The $\NP$-hardness of approximation algorithms reduces to that of gap problems as shown in the below observation:

\begin{theorem} \label{GapCSPtoAlgHard}
Suppose $\GapMaxCSP{\alpha}{\beta}$ is $\NP$-hard for CSP $\Psi$, then approximating $\MaxCSP$ to at least an $\frac{\alpha}{\beta}$ factor is $\NP$-hard.
\end{theorem}
%
\begin{proof}
Suppose there exists an algorithm $A$ which can $\frac{\alpha}{\beta}$-approximate $\MaxCSP$. For an instance $\I$ such that $\mathsf{Opt}(\I) \geq \beta$:

\begin{gather*}
A(\I) \geq  \frac{\alpha}{\beta} \cdot \mathsf{Opt}(\I) = \frac{\alpha}{\beta} \cdot \beta =  \alpha \\
\end{gather*}

Else if $\mathsf{Opt}(\I) < \alpha$
\begin{gather*}
A(\I) \leq \mathsf{Opt}(\I) < \alpha
\end{gather*}
by Definition \ref{Opt3SATDef} adapted to $\MaxCSP$ instances. Hence, the algorithm can decide  $\GapMaxCSP{\alpha}{\beta}$ by checking it's outputted value in respect to $\alpha$.
\end{proof}
%
This in particular demonstrates that showing the hardness of approximating a particular problem is equivalent to showing the hardness of its corresponding gap problem.

%It was shown early on that traditional Karp reductions were unlikely to produce the gaps required in Definition \ref{gapDef} in reducing $\SAT$ to such a promise problem.


%Traditional Karp reductions don't work
