Reachable set computation is an important technique for the verification of safety properties of dynamical systems.
%
% one method proposed for solving this problem, based on parallelotope bundle reachability for
%
In this paper, we investigate reachable set computation for discrete nonlinear systems based on parallelotope bundles.
%
The algorithm relies on computing an upper bound on the supremum of a nonlinear function over a rectangular domain, which has been traditionally done using Bernstein polynomials.
%or nonlinear optimization tools.
 We strive to remove the manual step of parallelotope template selection to make the method fully automatic.
%
Furthermore, we show that changing templates dynamically during computations can improve accuracy.
%
% One of the simplest algorithms for computing reachable sets for discrete nonlinear systems uses parallelotope bundles and Bernstein polynomials.
%
%Our main hypothesis in this paper is that generating templates in a dynamic manner would improve the accuracy of the reachable set.
%
To this end, we investigate two techniques for generating the template directions.
%
The first technique approximates the dynamics as a linear transformation and generates templates using this linear transformation.
%
The second technique uses Principal Component Analysis (PCA) of sample trajectories for generating templates.
%
We have implemented our approach in a Python-based tool called Kaa and improve its performance by two main enhancements.
%
% have implemented two main enhancements to it.
%
The tool is modular and use two types of global optimization solvers, the first using Bernstein polynomials and the second using NASA's Kodiak nonlinear optimization library.
%
Second, we leverage the natural parallelism of the reachability algorithm and parallelize the Kaa implementation.
% thus leveraging the cloud computing resources.
%
We demonstrate the improved accuracy of our approach on several standard nonlinear benchmark systems.
