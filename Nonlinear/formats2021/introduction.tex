
\section{Introduction}
\label{sec:intro}

One of the most widely-used techniques for performing safety analysis of nonlinear dynamical systems is reachable set computation.
%
The reachable set is defined to be the set of states visited by at least one of the trajectories of the system starting from an initial set.
%
Computing the reachable set for nonlinear systems is challenging primarily due to two reasons:
%
First, the tools for performing nonlinear analysis are not very scalable.
%
Second, computing the reachable set using set representations involves wrapping error.
%
That is, the overapproximation acquired at a given step would increase the conservativeness of the overapproximation for all future steps.

One of the techniques for computing the overapproximation of reachable sets for discrete time nonlinear systems is to use parallelotope bundles.
%
Here, the reachable set is represented as a parallelotope bundle, an intersection of several parallelotopes.
%
One of the advantages of this technique is its utilization of a special form of nonlinear optimization problem to overapproximate the reachable set.
%
The usage of a specific form of nonlinear optimization mitigates the drawback involved with the scalability of nonlinear analysis.

However, wrapping error still remains to be a problem for reachability using parallelotope bundles.
%
The template directions for specifying these parallelotopes are provided as an input by the user.
%
Often, these template directions are selected to be either the cardinal axis directions or some directions from octahedral domains.
%
However, it is not clear that the axis directions and octagonal directions are optimal for computing reachable sets.
%
Also, even an expert user of reachable set computation tools may not be able to provide a suitable set of template directions for computing reasonably accurate over-approximations of the reachable set.
%
Picking unsuitable template directions would only cause the wrapping error to increase, thus increasing the conservativeness of the safety analysis.

In this paper, we investigate techniques for generating template directions automatically and dynamically.
%
That is, instead of providing the template directions to compute the parallelotope, the user just specifies the number of templates and the algorithm automatically generates the template directions.
%
We study two techniques for generating the template directions.
%
First, we compute a local linear approximation of the nonlinear dynamics and use the linear approximation to compute the templates.
%
Second, we generate a specific set of sample trajectories from the set and use principal component analysis (PCA) over these trajectories.
%
We observe that the accuracy of the reachable set can be drastically improved by using templates generated using these two techniques.
%
For standard nonlinear benchmark systems, we show that generating templates in a dynamic fashion improves the accuracy of the reachable set by two orders of magnitude.
%
We demonstrate that even when the size of the initial set increases, our template generation technique returns more accurate reachable sets than both manually-specified and random template directions.

% [Introduction para]
% \begin{itemize}
% \item Reachable set computation.
% \item Nonlinear dynamics is challenging.
% \item Overapproximation, wrapping error.
% \item Parallelotope bundle reachability.
% \end{itemize}

% [Templates for reachability]
% \begin{itemize}
% \item Often the templates are generated statically.
% \item However, the most appropriate directions for templates that improve the accuracy is unknwon.
% \item Challenges: Wrapping error, cannot predict.
% \item Static template directions often aggrevate such errors.
% \item Generating templates dynamically is a challenge.
% \end{itemize}

% [Template generation]
% \begin{itemize}
% \item Template based overapproximation are used extensively in verification.
% \item More on this in the related work section.
% \item Dynamic templates using PCA have been investigated already.
% \item Not explicitly in the context of parallelotope bundles.
% \item In this paper we propose two techniques for  generating templates dynamically.
% \end{itemize}

\section{Related Work}

Reachable set computation of nonlinear systems using template polyhedra and Bernstein polynomials has been first proposed in~\cite{dang2009image}.
%
In~\cite{dang2009image}, Bernstein polynomial representation is used to compute an upper bound of a special type of nonlinear optimization problem~\cite{garloff2003bernstein}.
%
Several improvements to this algorithm were suggested in~\cite{dang2012reachability,sassi2012reachability} and~\cite{dang2014parameter} extends it for performing parameter synthesis.
%
The representation of parallelotope bundles for reachability was proposed in~\cite{dreossi2016parallelotope} and the effectiveness of using bundles for reachability was demonstrated in~\cite{dreossi2017sapo,dreossi2017reachability}.
%
However, all of these papers used static template directions for computing the reachable set.

Using template directions for reachable set has been proposed in~\cite{sankaranarayanan2008symbolic} and later improved in~\cite{dang2011template}.
%
Leveraging the principal component analysis of sample trajectories for computing reachable set has been proposed in~\cite{stursberg2003efficient,chen2011choice,seladji2017finding}.
%
More recently, connections between optimal template directions for reachability of linear dynamical systems and bilinear programming have been highlighted in~\cite{gronski2019template}.
%
For static template directions, octahedral domain directions~\cite{clariso2004octahedron} remain a popular choice.
