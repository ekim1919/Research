\chapter{Conclusion}
\label{chap:conclusion}
In this thesis, we covered a specific technique to perform reachability analysis over non-linear dynamical systems through leveraging parallelotope bundles. A particular non-linear optimization problem determines the offset of template directions required to express the reachable set.
%
We presented the original algorithm utilizing static template directions.
%
We then investigated two techniques for generating templates \emph{dynamically} and \emph{automatically}: the first using linear approximation of the dynamics, and the second using PCA.
%
We demonstrated that these techniques improve the accuracy of the reachable set of several benchmarks by an order of magnitude when compared to static or random template directions.
%
Finally, we experimented with potential applications of the Bernstein expansion technique to real-time reachability. We found that several hurdles and barriers arise to effectively utilizing the pruning method developed in the pre-processing phase.
%
Several remarks on ideas towards improvement of the methods we developed are listed below:
\begin{enumerate}
  \item Koopman linearization techniques for computing alternative linear approximation template directions with other optimization methods could yield interesting modifications to Algorithms \ref{alg:old} and \ref{alg:new} \cite{bak2021reachability}.

  \item The use of a massively-parallel implementation using HPC hardware, such as GPUs, for optimizing over an extremely large number of parallelotopes and their template directions is also an immediate extension. This is inspired by the approach behind the recent tool \emph{PIRK} \cite{devonport2020pirk}.

  \item Simply taking the intervals which dominate all the other intervals may be too crude to properly yield an effective speed-up. Are there any other methods of adding basis monomials to the lookup table which take into account the degree discrepancies outlined in Section \ref{sec:deg_disc}?
\end{enumerate}

%
