\chapter{Introduction}


% Introduction from previous works, sources (ARCH, Formats)_
% Related works from same papers.


One of the most widely-used techniques for performing safety analysis of non-linear dynamical systems is reachable set computation
%
For example, reachability analysis has found many applications in formally verifying the safety properties of Cyber-physical Systems governed by Neural Network Controllers \cite{tran2019star, fan2020reachnn, bak2021nnenum}.
%
The reachable set is defined to be the set of states visited by at least one of the trajectories of the system starting from an initial set and propagated forward in time by a finite fixed number of steps.
%
Computing the exact reachable set for non-linear systems is challenging due to several reasons:
%
First, unlike linear dynamical systems whose solutions can be expressed as closed form, non-linear dynamical systems generally do not admit such a nice form.
%
Second, computationally speaking, current tools for performing non-linear reachability analysis are not very scalable. This is also in stark contrast to several scalable approaches developed for linear dynamical systems \cite{duggirala2016parsimonious, bak2017simulation}.
%
Finally, computing the reachable set using various set representations involves wrapping error which may be too conservative for practical use.
%
That is, the overapproximation acquired at a given step would increase the conservativeness of the overapproximation for all future steps.

One of the several techniques for computing the overapproximation of reachable sets for discrete non-linear systems is to encode the reachable set through parallelotope bundles.
%
Here, the reachable set is represented as a parallelotope bundle, an geometric data structure representing an intersection of several simpler objects called parallelotopes.
%
One of the advantages of this technique is its exploitation of a special form of non-linear optimization problem to overapproximate the reachable set.
%
The usage of a specific form of non-linear optimization mitigates many drawbacks involved with the scalability of non-linear analysis.

However, wrapping error still remains to be a problem for reachability using parallelotope bundles.
%
An immediate reason stems from the responsbility of the practitioner to define the template directions specifiying the parallelotopes.
%
Often, these template directions are selected to be either the cardinal axis directions or some directions from octahedral domains.
%
However, it is not certain that the axis-aligned and octagonal directions are optimal for computing reachable sets over general non-linear dynamics.
%
Additionally, even an expert user of reachable set computation tools may not be able to ascertain a suitable set of template directions for computing reasonably accurate over-approximations of the reachable set.
%
Picking unsuitable template directions would only cause the wrapping error to grow, leading to the aforementioned issue of overly conservative reachable sets.

In this thesis, we investigate techniques for generating template directions automatically and dynamically, which is the culmination of several publications in different venues \cite{kim2020kaa,kim2021automatic,geretti2021arch}.
%
Specifically, we propose a method where instead of the user providing the template directions to define the parallelotope bundle, he or she specifies the number of templates whose template directions are to be generated by our algorithm automatically.

To this end, we study two techniques for generating the said template directions.
%
First, we compute a local linear approximation of the non-linear dynamics and use the linear approximation to compute the template directions.
%
Second, we generate a set of trajectories sampled from within the reachable set and use Principal Component Analysis (PCA) over these trajectories.
%
We observe that the accuracy of the reachable set can be drastically improved by using templates generated using these two techniques.
%
To address scalability, we demonstrate that even when the size of the initial set increases, our template generation algorithm returns more accurate reachable sets than both manually-specified and random template directions.
%
Finally, we experiment with our dynamic template generation algorithm's effectiveness on approximating the reachable set of high-dimensional COVID19 dynamics proposed by the Indian Supermodel Committee \cite{indiansuper2020supermodel}. The results were published in an ACM blogpost detailing the utility of reachable set computation in modeling disease dynamics \cite{bak2021covid}.

% [Introduction para]asdf
% \begin{itemize}
% \item Reachable set computation.
% \item non-linear dynamics is challenging.
% \item Overapproximation, wrapping error.
% \item Parallelotope bundle reachability.
% \end{itemize}

% [Templates for reachability]
% \begin{itemize}
% \item Often the templates are generated statically.
% \item However, the most appropriate directions for templates that improve the accuracy is unknwon.
% \item Challenges: Wrapping error, cannot predict.
% \item Static template directions often aggrevate such errors.
% \item Generating templates dynamically is a challenge.
% \end{itemize}

% [Template generation]
% \begin{itemize}
% \item Template based overapproximation are used extensively in verification.
% \item More on this in the related work section.
% \item Dynamic templates using PCA have been investigated already.
% \item Not explicitly in the context of parallelotope bundles.
% \item In this paper we propose two techniques for  generating templates dynamically.
% \end{itemize}

\section{Related Work}
\label{sec:related}

Reachable set computation of non-linear systems using template polyhedra and Bernstein polynomials has been first proposed in~\cite{dang2009image}.
%
In~\cite{dang2009image}, Bernstein polynomial representation is used to compute an upper bound of a special type of non-linear optimization problem.
%
This enclosing property of Bernstein polynomials has been actively studied in the area of global optimization \cite{nataray2002algorithm, garloff2003bernstein, nataraj2007new}.
%
Furthermore, several heuristics have been proposed for improving the computational performance of optimization using Bernstein polynomials~\cite{smith2009fast,munoz2013formalization}.

Several improvements to this algorithm were suggested in~\cite{dang2012reachability, sassi2012reachability} and~\cite{dang2014parameter} extends it for performing parameter synthesis.
%Hey
The representation of parallelotope bundles for reachability was proposed in~\cite{dreossi2016parallelotope} and the effectiveness of using bundles for reachability was demonstrated in~\cite{dreossi2017sapo, dreossi2017reachability}.
%
However, all of these papers used static template directions for computing the reachable set.
%
In other words, the user must specify the template directions before the reachable set computation proceeds.

Using template directions for reachable set has been proposed in~\cite{sankaranarayanan2008symbolic} and later improved in~\cite{dang2011template}.
%
Leveraging the principal component analysis of sample trajectories for computing reachable set has been proposed in~\cite{stursberg2003efficient,chen2011choice,seladji2017finding}.
%
More recently, connections between optimal template directions for reachability of linear dynamical systems and bilinear programming have been highlighted in~\cite{gronski2019template}.
%
For static template directions, octahedral domain directions~\cite{clariso2004octahedron} remain a popular choice.
