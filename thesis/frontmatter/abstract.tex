%The word �Abstract� should be centered 2? below the top of the page.
%Skip one line, then center your name followed by the title of the
%thesis/dissertation. Use as many lines as necessary. Centered below the
%title include the phrase, in parentheses, �(Under the direction of
%_________)� and include the name(s) of the dissertation advisor(s).
%Skip one line and begin the content of the abstract. It should be
%double-spaced and conform to margin guidelines. An abstract should not
%exceed 150 words for a thesis and 350 words for a dissertation. The
%latter is a requirement of both the Graduate School and UMI's
%Dissertation Abstracts International.
%Because your dissertation abstract will be published, please prepare and
%proofread it carefully. Print all symbols and foreign words clearly and
%accurately to avoid errors or delays. Make sure that the title given at
%the top of the abstract has the same wording as the title shown on your
%title page. Avoid mathematical formulas, diagrams, and other
%illustrative materials, and only offer the briefest possible description
%of your thesis/dissertation and a concise summary of its conclusions. Do
%not include lengthy explanations and opinions.
%The abstract should bear the lower case Roman number ii (if you did not
%include a copyright page) or iii (if you include a copyright page).

\begin{center}
\vspace*{52pt}
{\Large \textbf{ABSTRACT}}
\vspace{11pt}

\begin{singlespace}
Edward Kim: \thesistitle \\
(Under the direction of Parasara Sridhar Duggirala)
\end{singlespace}

\end{center}
\par Reachable set computation is an important technique for the verification of safety properties of dynamical systems.
%
% one method proposed for solving this problem, based on parallelotope bundle reachability for
%
In this thesis, we investigate reachable set computation for discrete nonlinear systems based on parallelotope bundles.
%
The crux of the reachability algorithm relies on computing an upper and lower bound on the supremum and infimum respectively of a nonlinear function over a rectangular domain.
%
We cover two ways of computing these bounds: one method utilizing Bernstein polynomials and the other relying on a non-linear optimization tool developed by NASA, Kodiak.
%
We aim to improve the traditional parallelotope-based reachability method by removing the manual step of parallelotope template selection in order to make the procedure fully automatic.
%
Furthermore, we show that adding templates dynamically during computations can improve accuracy.
%
% One of the simplest algorithms for computing reachable sets for discrete nonlinear systems uses parallelotope bundles and Bernstein polynomials.
%
%Our main hypothesis in this paper is that generating templates in a dynamic manner would improve the accuracy of the reachable set.
%
To this end, we investigate two techniques for generating the template directions.
%
The first technique approximates the dynamics as a linear transformation and generates templates using this linear transformation.
%
The second technique uses Principal Component Analysis (PCA) of sample trajectories for generating templates.
%
We have implemented our approach in a Python-based tool called Kaa.
%
%Kaa was developed as a more compact, simple implementation of the pre-existing parallelotope-based reachability tool, Sapo.
%
% have implemented two main enhancements to it.
%
The tool is modular and use two types of global optimization solvers, the first using Bernstein polynomials and the second using
the aforementioned Kodiak library.
%
Additionally, we leverage the natural parallelism of the reachability algorithm and parallelize the Kaa implementation.
%
Finally, we demonstrate the improved accuracy of our approach on several standard nonlinear benchmark systems, including a high-dimensional COVID19 model proposed by the Indian Supermodel Committee.


\clearpage
