%The word �Abstract� should be centered 2? below the top of the page.
%Skip one line, then center your name followed by the title of the
%thesis/dissertation. Use as many lines as necessary. Centered below the
%title include the phrase, in parentheses, �(Under the direction of
%_________)� and include the name(s) of the dissertation advisor(s).
%Skip one line and begin the content of the abstract. It should be
%double-spaced and conform to margin guidelines. An abstract should not
%exceed 150 words for a thesis and 350 words for a dissertation. The
%latter is a requirement of both the Graduate School and UMI's
%Dissertation Abstracts International.
%Because your dissertation abstract will be published, please prepare and
%proofread it carefully. Print all symbols and foreign words clearly and
%accurately to avoid errors or delays. Make sure that the title given at
%the top of the abstract has the same wording as the title shown on your
%title page. Avoid mathematical formulas, diagrams, and other
%illustrative materials, and only offer the briefest possible description
%of your thesis/dissertation and a concise summary of its conclusions. Do
%not include lengthy explanations and opinions.
%The abstract should bear the lower case Roman number ii (if you did not
%include a copyright page) or iii (if you include a copyright page).

\begin{center}
\vspace*{52pt}
{\Large \textbf{ABSTRACT}}
\vspace{11pt}

\begin{singlespace}
Edward Kim: \thesistitle \\
(Under the direction of Parasara Sridhar Duggirala)
\end{singlespace}

\end{center}
\par Reachable set computation is an important technique for the verification of safety properties of dynamical systems.
%
% one method proposed for solving this problem, based on parallelotope bundle reachability for
%
In this thesis, we investigate reachable set computation for discrete non-linear systems based on parallelotope bundles.
%
The crux of the reachability algorithm relies on computing an upper and lower bound on the supremum and infimum respectively of a non-linear function over a rectangular domain.
%
Bernstein Expansion of a polynomial function has been explored as a traditional method for computing these bounds effciently.
%
In light of this, we aim to improve the traditional parallelotope-based reachability method by removing the manual step of parallelotope template selection in order to make the procedure fully automatic.
%
Furthermore, we show that adding templates dynamically during computations can improve accuracy.
%
To this end, we investigate two techniques for generating template directions.
%
The first technique approximates the dynamics as a linear transformation and generates templates using this transformation.
%
The second technique uses Principal Component Analysis (PCA) of sample trajectories for generating templates.
%
We have implemented our approach in a Python-based tool called Kaa, which uses two types of global optimization solvers, the first using Bernstein polynomials and the second using
the Kodiak library.
%
We demonstrate the improved accuracy of our approach on several standard nonlinear benchmark systems, including a high-dimensional COVID19 model.
%
Finally, we explore a potential application of the Bernstein expansion technique to real-time reachability.
%
We present evidence of several hurdles and barriers against effectively utilizing our Bernstein coefficient pruning method.
\clearpage
